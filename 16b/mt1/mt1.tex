\documentclass[]{article}
\usepackage{amsmath, amsfonts, amssymb, amsthm}
\usepackage{cancel}
\usepackage{pgfplots}
\usepackage{siunitx}
\usepackage[american]{circuitikz}
\usepackage{bm}
\usepackage{graphicx}
\usepackage{fullpage}
\usepackage{pdfpages}

\renewcommand{\thesection}{\arabic{section}}
\renewcommand{\thesubsection}{\thesection.\alph{subsection}}
\renewcommand{\thesubsubsection}{\thesubsection.\roman{subsubsection}}

\newtheorem{genthm}{Theorem}

\newcommand{\unit}[1]{\bm{\hat{#1}}}
\newcommand{\iprod}[2]{\left\langle #1, #2 \right\rangle}
\newcommand{\tpose}[1]{\left[#1\right]^{\! \top} \!\!}
\newcommand{\diff}[1]{\frac{d}{d #1}}

\title{EECS 16B MT1 Redo}
\author{Bryan Ngo}
\date{2020-03-12}

\begin{document}

\maketitle

\section{CMOS Circuits}

\begin{equation}
	\begin{array}{c|c|l}
	V_{in, 1} & V_{in, 2} & V_o \\
	\hline
	\SI{0}{\volt} & \SI{0}{\volt} & V_{DD} \\
	V_{DD} & 0 & 0 \\
	0 & V_{DD} & 0 \\
	V_{DD} & V_{DD} & 0
	\end{array}
\end{equation}

\section{Differential Equations}

\subsection{}

\begin{align}
	\diff{t} M(t) &= -r M(t) \\
	\int \frac{1}{M(t)} \, dM &= \int -r \, dt \\
	\ln|M(t)| &= -rt + C \\
	M(t) &= Ce^{-rt} \\
	M(0) &= M_0 = C \\
	\Rightarrow M(t) &= M_0 e^{-rt}
\end{align}

\subsection{}

\begin{align}
	\cancel{M_0} e^{-rt} &= \frac{1}{2} \cancel{M_0} \\
	-rt &= \ln\left(\frac{1}{2}\right) = -\ln(2) \\
	t &= \frac{\ln(2)}{r}
\end{align}

\section{Complex Numbers}

\subsection{}

\subsubsection{}

\begin{align}
	|1 + j| &= \sqrt{2} \\
	\arg(1 + j) &= \frac{\pi}{4} \\
	\Rightarrow 1 + j &= \sqrt{2} e^{j \frac{\pi}{4}}
\end{align}

\subsubsection{}

\begin{equation}
	\sqrt{j} = \sqrt{e^{j \frac{\pi}{2}}} = e^{j \frac{\pi}{4}}
\end{equation}

\subsection{}

\subsubsection{}

\begin{equation}
	3e^{j \frac{\pi}{3}} = 3\left(\cos\left(\frac{\pi}{3}\right) + j \sin\left(\frac{\pi}{3}\right)\right) = \frac{3}{2} + j \frac{3\sqrt{3}}{2}
\end{equation}

\subsubsection{}

\begin{equation}
	-\sqrt{7} e^{j \pi} = \sqrt{7}
\end{equation}

\subsection{}

\subsubsection{}

\begin{proof}
\begin{equation}
	\frac{1}{j} \cdot \frac{j}{j} = \frac{j}{j^2} = -j
\end{equation}
\end{proof}

\subsubsection{}

\begin{proof}
\begin{align}
	\sin(2x) &= \frac{e^{2x} - e^{-2x}}{2j} =  \frac{(e^x)^2 - (e^{-x})^2}{2j} \\
	&= \frac{(\cos{x} + j \sin{x})^2 - (\cos{x} - j \sin{x})^2}{2j} \\
	&= \frac{\cancel{\cos^2(x)} + 2j \cos(x) \sin(x) - \cancel{\sin^2(x)} - \cancel{\cos^2(x)} + 2j \cos(x) \sin(x) + \cancel{\sin^2(x)}}{2j} \\
	&= \frac{4j \cos(x) \sin(x)}{2j} = 2 \cos(x) \sin(x)
\end{align}
\end{proof}

\section{Vector Differential Equation}

\subsection{}

To derive \(\bm{D}\),
\begin{align}
	\diff{t} \bm{z} &= \bm{Dz} \\
	\diff{t} \bm{Tx} &= \bm{DTx} \\
	\diff{t} \bm{x} &= \underbrace{\bm{T}^{-1}\bm{DT}}_{\bm{A}} \bm{x} \\
	\Rightarrow \bm{D} &= \bm{TA}\bm{T}^{-1}
\end{align}
where \(\bm{T} = \bm{V}^{-1} = \begin{bmatrix}
\bm{v}_1 & \bm{v}_2
\end{bmatrix}^{-1}\).
Thus,
\begin{align}
	\bm{T} &=
	\begin{bmatrix}
	j & -j \\
	1 & 1
	\end{bmatrix}^{-1} =
	\frac{1}{2j} \begin{bmatrix}
	1 & j \\
	-1 & j
	\end{bmatrix} \\
	\bm{D} = \bm{TA}\bm{T}^{-1} &= \frac{1}{2j} \begin{bmatrix}
	1 & j \\
	-1 & j
	\end{bmatrix}
	\begin{bmatrix}
	\alpha & -\omega \\
	\omega & \alpha
	\end{bmatrix}
	\begin{bmatrix}
	j & -j \\
	1 & 1
	\end{bmatrix} \\
	&= \frac{1}{2j} \begin{bmatrix}
	1 & j \\
	-1 & j
	\end{bmatrix}
	\begin{bmatrix}
	j\alpha - \omega & -j\alpha - \omega \\
	j\omega + \alpha & -j\omega + \alpha
	\end{bmatrix} \\
	&= \frac{1}{2j} \begin{bmatrix}
	j\alpha - \omega - \omega + j\alpha & \cancel{-j\alpha - \omega + \omega + j\alpha} \\
	\cancel{-j\alpha + \omega - \omega + j\alpha} & j\alpha + \omega + \omega + j\alpha
	\end{bmatrix} \\
	&= -\frac{j}{2} \begin{bmatrix}
	2j\alpha - 2\omega & 0 \\
	0 & 2j\alpha + 2\omega
	\end{bmatrix} = \begin{bmatrix}
	\alpha + j\omega & 0 \\
	0 & \alpha - j\omega
	\end{bmatrix}
\end{align}

\subsection{}

\begin{align}
	\diff{t} \bm{z} &= \begin{bmatrix}
	\alpha + j\omega & 0 \\
	0 & \alpha - j\omega
	\end{bmatrix} \bm{z} \\
	\Rightarrow \bm{z} &= \begin{bmatrix}
	z_a e^{(\alpha + j\omega)t} \\
	z_b e^{(\alpha - j\omega)t}
	\end{bmatrix} \\
	\bm{z}(0) = \bm{T}\bm{x}(0) &= -\frac{j}{2} \begin{bmatrix}
	1 & j \\
	-1 & j
	\end{bmatrix}
	\begin{bmatrix}
	1 \\
	0
	\end{bmatrix} = \begin{bmatrix}
		-\frac{j}{2} \\
		\frac{j}{2}
	\end{bmatrix}
\end{align}

\subsection{}

\begin{align}
	\bm{x} = \bm{T}^{-1} \bm{z} &= \begin{bmatrix}
	j & -j \\
	1 & 1
	\end{bmatrix}
	\begin{bmatrix}
	-\frac{j}{2} e^{(\alpha + j\omega)t} \\
	\frac{j}{2} e^{(\alpha - j\omega)t}
	\end{bmatrix} \\
	x_1 &= \frac{1}{2} e^{(\alpha + j\omega)t} + \frac{1}{2} e^{(\alpha - j\omega)t} \\
	&= e^{\alpha t} \left(\frac{e^{j \omega t} + e^{-j \omega t}}{2}\right) \\
	&= e^{\alpha t} \cos(\omega t)
\end{align}

\section{Transient Identification}

\subsection{\(v(t) = V_0 e^{\lambda t}, \lambda < 0\)}

RC, RCRC. RL

\subsection{\(v(t) = V_0 e^{jt}\)}

None

\subsection{\(v(t) = V_0 e^{\alpha t} \cos(\omega t), \alpha < 0\)}

Underdamped RLC

\subsection{\(v(t) = V_1 e^{\alpha t} + V_2 e^{\beta t}; \ \alpha, \beta < 0\)}

RCRC, overdamped RLC

\section{Phasors}

\begin{center}
\begin{tabular}{c||l|l}
	& \textbf{Phasor} & \textbf{Waveform} \\
	\hline
	(a) & \(e^{j \frac{\pi}{2}}\) & 5 \\
	(b) & \(3e^{j \frac{\pi}{2}}\) & 6 \\
	(c) & \(1\) & 3 \\
	(d) & \(e^{-j \frac{\pi}{2}}\) & 1 \\
	(e) & \(e^{-j \pi}\) & 2 \\
	(f) & \(e^{j \cdot 0} + e^{j \frac{\pi}{2}}\) & 4
\end{tabular}
\end{center}

\section{Energy}

\subsection{}

\begin{center}
\begin{tikzpicture}
	\begin{axis}[
		width = \textwidth,
		height = \axisdefaultheight,
		xlabel = \(t\),
		ylabel = \(V_C\),
		xmin = 0, xmax = 7,
		ymin = 0.8, ymax = 1.2,
		xtick = {0}, ytick = {0},
		extra tick style = {grid=major},
		extra x ticks = {1, 3, 4, 6, 7},
		extra y ticks = {0.9, 1},
		extra x tick labels = {\(t_1\), \(t_1 + T_{on}\), \(t_2\), \(t_2 + T_{on}\), \(T\)},
		extra y tick labels = {\SI{0.9}{\volt}, \SI{1}{\volt}}
	]
		\addplot[domain=0:1, color=blue] {0.9};
		\addplot[domain=1:3, color=red] {1 - 0.1*exp(-3*(x - 1))};
		\addplot[domain=3:4, color=blue] {1};
		\addplot[domain=4:6, color=red] {0.1*exp(-3*(x - 4)) + 0.9};
		\addplot[domain=6:7, color=blue] {0.9};
	\end{axis}
\end{tikzpicture}
\end{center}

\subsection{}

Since the \SI{1.0}{\volt} source is only connected during the time \(t \in [t_1, t_1 + T_{on}]\), we only need to consider the change in voltage across those points.
Using the definition of capacitance,
\begin{equation}
	Q = C \Delta V_c = C (1.0 - 0.9) = 0.1 C \, \si{\coulomb}
\end{equation}

\subsection{}

Since at any given time \emph{at most} only one transistor with equivalent gate resistance is connected in series with the capacitor, the energy dissipated across the entire circuit is simply the change in energy across the capacitor from \(t \in [0, T]\).
Since the change in energy of a capacitor is only dependent on voltage, we only concern ourselves with whenever the voltage changes, i.e. \(t \in [t_1, t_1 + T_{on}] \cup [t_2, t_2 + T_{on}]\),
\begin{align}
	\Delta E_{tot} &= |\left.\Delta E\right|_{t_1}^{t_1 + T_{on}} + \left.\Delta E\right|_{t_2}^{t_2 + T_{on}}| \\
	&= \frac{1}{2} C(V(t_1 + T_{on})^2 - V(t_1)^2) + \frac{1}{2} C(V(t_2)^2 - V(t_2 + T_{on})^2) \\
	&= C (1^2 - 0.9^2) = 0.19C \, \si{\joule}
\end{align}

\newpage

%\includepdf[pages=-]{prob*.pdf}

\end{document}
