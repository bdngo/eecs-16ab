\documentclass[]{article}
\usepackage{amsmath, amsfonts, amssymb, amsthm}
\usepackage{cancel}
\usepackage{pgfplots}
\usepackage{siunitx}
\usepackage[american]{circuitikz}
\usepackage{bm}
\usepackage{graphicx}
\usepackage{fullpage}
\usepackage{pdfpages}

\renewcommand{\thesection}{\arabic{section}}
\renewcommand{\thesubsection}{\thesection.\alph{subsection}}
\renewcommand{\thesubsubsection}{\thesubsection.\roman{subsubsection}}

\newtheorem{genthm}{Theorem}
\newcommand{\unit}[1]{\bm{\hat{#1}}}
\newcommand{\iprod}[2]{\left\langle #1, #2 \right\rangle}
\newcommand{\tpose}[1]{\left[#1\right]^{\! \top} \!\!}
\newcommand{\diff}[1]{\frac{d}{d #1}}
\newcommand{\Span}{\operatorname{span}}

\title{EECS 16B HW07}
\author{Bryan Ngo}
\date{2020-03-16}

\begin{document}

\maketitle

\section{LED Strip}

\subsection{}

We can use a vector \(\bm{x}\) that represents the brightness of all the LEDs,
\begin{equation}
	\bm{x} = \begin{bmatrix}
	x_1 \\
	x_2 \\
	x_3 \\
	x_4 \\
	x_5
	\end{bmatrix}
\end{equation}
where our input is the brightness of the left-most LED.
We will assume leftmost to be the brightness \(x_1\).

\subsection{}

The system can be written is
\begin{equation}
	\bm{x}[t + 1] =
	\begin{bmatrix}
	0 & 0 & 0 & 0 & 0 \\
	1 & 0 & 0 & 0 & 0 \\
	0 & 1 & 0 & 0 & 0 \\
	0 & 0 & 1 & 0 & 0 \\
	0 & 0 & 0 & 1 & 0
	\end{bmatrix} \bm{x}
	+
	\begin{bmatrix}
	1 \\
	0 \\
	0 \\
	0 \\
	0
	\end{bmatrix} u(t)
\end{equation}

\subsection{}

\begin{proof}
To prove controllability, we must prove that
\begin{equation}
	\Span\{\bm{u}, \bm{A}\bm{u}, \bm{A}^2 \bm{u}, \ldots, \bm{A}^{n - 1} \bm{u}\} = \mathbb{R}^n
\end{equation}
The system is controllable because every value of the state vector can be influenced by the input.
This is because we have a sort of "daisy chain" between the LEDs, which means every brightness can be traced back to the input, with a certain delay.
That is, we can change the brightness of any given LED at any time.
\end{proof}

\subsection{}

\begin{equation}
	x[0] = \begin{bmatrix}
	0 \\
	127 \\
	0 \\
	255 \\
	0
	\end{bmatrix}
\end{equation}
We can simply maintain this brightness by looping through the state vector's entries and pushing the next brightness.
By \(t = 5\), the brightness will be the same.
We can display any pattern by simply pushing the brightness one at a time.

\section{Controllability in Circuits}

\begin{center}
\begin{circuitikz}\draw
	(0, 0) to[V=\(V_s\), invert] (0, 2) to[C=\(C_1\), v=\(V_1\)] (2, 2) to[C=\(C_2\), v=\(V_2\), *-*] (4, 2) to[R=\(R\)] (4, 0) to[short] (0, 0)
;\end{circuitikz}
\end{center}

\subsection{}

Performing KCL at the node between the capacitors and across the resistor and using KVL,
\begin{align}
	V_s - V_1 - V_2 - V_R &= 0 \\
	C_2\diff{t} V_2 - C_1\diff{t} V_1 &= 0 \\
	\frac{V_s - V_1 - V_2}{R} - C_2\diff{t} V_2 &= 0
\end{align}
In state space form,
\begin{align}
	\diff{t} \begin{bmatrix}
	V_1 \\
	V_2
	\end{bmatrix} = \begin{bmatrix}
	\frac{C_2}{C_1} \diff{t} V_2 \\
	\frac{V_s - V_1 - V_2}{RC_2}
	\end{bmatrix} &= \begin{bmatrix}
	\frac{V_s - V_1 - V_2}{RC_1} \\
	\frac{V_s - V_1 - V_2}{RC_2}
	\end{bmatrix} \\
	&= \begin{bmatrix}
	-\frac{1}{RC_1} & -\frac{1}{RC_1} \\
	-\frac{1}{RC_2} & -\frac{1}{RC_2}
	\end{bmatrix} \begin{bmatrix}
	V_1 \\
	V_2
	\end{bmatrix} + \begin{bmatrix}
	\frac{1}{RC_1} \\
	\frac{1}{RC_2}
	\end{bmatrix} V_s
\end{align}

\subsection{}

\begin{proof}
At \(t = 2\), the span is
\begin{equation}
	\Span\left\{
	\begin{bmatrix}
	\frac{1}{RC_1} \\
	\frac{1}{RC_2}
	\end{bmatrix},
	\begin{bmatrix}
	-\frac{1}{R^2C_1^2} - \frac{1}{R^2C_1C_2} \\
	-\frac{1}{R^2C_2C_1} - \frac{1}{R^2C_2^2}
	\end{bmatrix}
	\right\} = \mathbb{R}
\end{equation}
If we focus on the \(\bm{Ab}\) vector,
\begin{equation}
	\begin{bmatrix}
	-\frac{1}{R^2C_1^2} - \frac{1}{R^2C_1C_2} \\
	-\frac{1}{R^2C_2C_1} - \frac{1}{R^2C_2^2}
	\end{bmatrix} = \begin{bmatrix}
	\frac{1}{RC_1} \left(-\frac{1}{RC_1} - \frac{1}{RC_2}\right) \\
	\frac{1}{RC_2} \left(-\frac{1}{RC_1} - \frac{1}{RC_2}\right)
	\end{bmatrix} = \left(-\frac{1}{RC_1} - \frac{1}{RC_2}\right) \begin{bmatrix}
	\frac{1}{RC_1} \\
	\frac{1}{RC_2}
	\end{bmatrix}
\end{equation}
Since \(\bm{Ab}\) is simply a scaled version of \(\bm{b}\), this means that the span does not increase from \(t = 1\) to \(t = 2\).
Thus, the span will never increase and remain at \(\mathbb{R}\) and is uncontrollable.
\end{proof}

\subsection{}

The system does not appear to be controllable because \(V_1\) is dependent on \(V_2\), as is seen in \textbf{Equation 8}.
This means that \(V_1\) cannot be independently controlled by \(V_s\), our input.

\subsection{}

\begin{center}
\begin{circuitikz}\draw
	(0, 0) to[V=\(V_s\), invert] (0, 2) to[C=\(C_1 \parallel C_2\), v=\(V_C\)] (3, 2) to[R=\(R\)] (3, 0) to[short] (0, 0)
;\end{circuitikz}
\end{center}
We can still control \(V_s\) in this system, but it is controllable now since every quantity is independently controllable via the input.

\section{Controllability and Discretization}

\begin{align}
	\diff{t} p(t) &= v(t) \\
	\diff{t} v(t) &= u(t)
\end{align}

\subsection{}

Converting to matrix form,
\begin{equation}
	\diff{t} \begin{bmatrix}
	p(t) \\
	v(t)
	\end{bmatrix}
	=
	\begin{bmatrix}
	0 & 1 \\
	0 & 0
	\end{bmatrix}
	\begin{bmatrix}
	p(t) \\
	v(t)
	\end{bmatrix}
	+
	\begin{bmatrix}
	0 \\
	1
	\end{bmatrix} u(t)
\end{equation}
The span at \(t = 2\) are
\begin{align}
	&\Span\left\{
	\begin{bmatrix}
	0 \\
	1
	\end{bmatrix},
	\begin{bmatrix}
	1 \\
	0
	\end{bmatrix}
	\right\} = \mathbb{R}^2
\end{align}
Since the two vectors are linearly independent (specifically, \(\unit{\imath}, \unit{\jmath}\)), the system spans the required space and is thus, controllable.

\subsection{}

Suppose we find the change in the state variables \(p(t), v(t)\) according to a change \(T\) in time,
\begin{align}
	p(t + T) - p(t) &= \int_t^{t + T} v(\tau) \, d\tau \\
	v(t + T) - v(t) &= \int_t^{t + T} u(\tau) \, d\tau = T u(t)
\end{align}
Substituting \(v(\tau) = v(t) + (\tau - t) u(t)\),
\begin{align}
	p(t + T) - p(t) &= \int_t^{t + T} v(t) + (\tau - t) u(t) \, d\tau \\
	&= T v(t) + \int_t^{t + T} (\tau - t) u(t) \, d\tau \\
	&= \left.T v(t) + \frac{(\tau - t)^2}{2} u(t)\right|_t^{t + T} \\
	&= T v(t) + \frac{T^2}{2} u(t)
\end{align}
In matrix form, our discretized state system is
\begin{equation}
	\begin{bmatrix}
	p[t + 1] \\
	v[t + t]
	\end{bmatrix}
	=
	\begin{bmatrix}
	1 & T \\
	0 & 1
	\end{bmatrix}
	\begin{bmatrix}
	p[t] \\
	v[t]
	\end{bmatrix}
	+
	\begin{bmatrix}
	\frac{T^2}{2} \\
	T
	\end{bmatrix} u[t]
\end{equation}

\subsection{}

Our span at \(t = 2\) is
\begin{equation}
	\Span\left\{
	\begin{bmatrix}
	\frac{T^2}{2} \\
	T
	\end{bmatrix},
	\begin{bmatrix}
	\frac{3T^2}{2} \\
	\frac{T^2}{2}
	\end{bmatrix}
	\right\} = \mathbb{R}^2
\end{equation}
By inspection, the two vectors are clearly able to span \(\mathbb{R}^2\), and thus the system is controllable.

\section{Understanding the SIXT33N Car Control Model}

\begin{equation}
	v[k] = d[k + 1] - d[k] = \theta u[k] - \beta
\end{equation}

\subsection{}

Setting \(v[k] = v^\ast\),
\begin{equation}
	u[k] = \frac{v^\ast + \beta}{\theta}
\end{equation}

\subsection{}

\begin{equation}
	v[k] =
	\begin{cases}
	-\beta & u[k] = 0 \\
	255\theta - \beta & u[k] = 255
	\end{cases}
\end{equation}
We are able to slow the car down by reducing \(u[k]\).

\subsection{}

At \(u[k] = 0\), \(v[k] = -\beta\).
This means that the car is supposedly going \emph{backwards}, contrary to our intuition, which should be at \(u[k] = 0\), \(v[k] = 0\).
The model \emph{cannot} accurately describe this phenomenon.

\subsection{}

We can determine \(\theta_L, \theta_R, \beta_L, \beta_R\) simply by applying a series of known inputs and an expected series of velocity outputs and using statistical techniques like least squares to create \(\theta, \beta\).

\subsection{}

Since the velocity curve of SIXT33N is nonlinear, we need to restrict our speeds to a very small range.
This makes the curve look linear around that range.
Then, since \(\theta_L, \theta_R, \beta_L, \beta_R\) are separate values, we can use different parameters for each wheel, thus counteracting any deviance from the model.

\stepcounter{section}

\section{Homework Process and Study Group}

\begin{enumerate}
	\item I used the Lecture 7B notes and Discussion 8A solutions.
	\item I worked on this homework by myself.
	\item I worked on this homework in one sitting.
	\item Perhaps a little more explanation on the work going through \textbf{3.b} since it was not discussed in lecture.
	\item 4 hours.
\end{enumerate}

\newpage

%\includepdf[pages=-]{prob*.pdf}

\end{document}
