\documentclass[]{article}
\usepackage{fullpage}
\usepackage{amsmath, amsfonts, amssymb, amsthm}
\usepackage{cancel}
\usepackage{siunitx}
\usepackage[american]{circuitikz}
\usepackage{pgfplots}
\usepackage{bm}
\usepackage{graphicx}
\usepackage{pdfpages}

\renewcommand{\thesection}{\arabic{section}}
\renewcommand{\thesubsection}{\thesection.\alph{subsection}}
\renewcommand{\thesubsubsection}{\thesubsection.\roman{subsubsection}}

\newtheorem{genthm}{Theorem}

\newcommand{\unit}[1]{\bm{\hat{#1}}}
\newcommand{\iprod}[2]{\left\langle #1, #2 \right\rangle}
\newcommand{\tpose}[1]{\left[#1\right]^{\! \top} \!\!}
\newcommand{\diff}[1]{\frac{d}{d #1}}

\title{EECS 16B HW02}
\author{Bryan Ngo}
\date{2020-01-25}

\begin{document}

\maketitle

\section{Fundamental Theorem of Solutions to Differential Equations}

\subsection{}

\begin{genthm}
Given \(\phi_1(x) = e^x\) and \(\phi_2(x) = \sum_{n = 0}^{\infty} \frac{x^n}{n!}\), \(\phi_1(x) = \phi_2(x)\).
\end{genthm}

\begin{proof}
Proving the derivative property,
\begin{align}
	\diff{x} \phi_1(x) &= \diff{x} e^x = e^x = \phi_1(x) \\
	\diff{x} \phi_2(x) &= \diff{x} \sum_{n = 0}^{\infty} \frac{x^n}{n!} = \sum_{n = 0}^{\infty} \frac{\cancel{n} x^{n - 1}}{\cancel{n} (n - 1)!} = \sum_{n = 1}^{\infty} \frac{x^{n - 1}}{(n - 1)!} = \phi_2(x)
\end{align}
where we are allowed to shift the index because any singular \(n\) terms cancel to zero.
Proving the initial condition,
\begin{align}
	\phi_1(0) &= e^0 = 1 \\
	\phi_2(0) &= \sum_{n = 0}^{\infty} \frac{0^n}{n!} = \frac{0^0}{0!} + \cancel{\frac{0^1}{1!}} + \cancel{\frac{0^2}{2!}} + \cancel{\cdots} = 1
\end{align}
\end{proof}

\subsection{}

\begin{genthm}
Given \(\phi_1(x) = \cos(x)\) and \(\phi_2(x) = \cos(-x)\), \(\phi_1(x) = \phi_2(x)\).
\end{genthm}

\begin{proof}
Proving the derivatives,
\begin{align}
	\frac{d^2}{dx^2} \phi_1(x) &= \frac{d^2}{dx^2} \cos(x) = -\cos(x) = -\phi_1(x) \\
	\frac{d^2}{dx^2} \phi_2(x) &= \frac{d^2}{dx^2} \cos(-x) =  -\cos(-x) = -\phi_2(x)
\end{align}
Proving the initial conditions,
\begin{align}
	\phi_1(x) &= \phi_2(x) = \cos(0) = 1 \\
	\diff{x} \phi_1(0) &= -\sin(0) = 0 \\
	\diff{x} \phi_2(0) &= \sin(0) = 0
\end{align}
\end{proof}

\section{IC Power Supply}

% TODO: FINISH

\subsection{}

\begin{center}
\begin{tikzpicture}
	\begin{axis}[
		xlabel = {\(t [\si{\nano\second}]\)},
		ylabel = {\(V_{out1}(t) [\si{\volt}]\)},
		xmin = 0, xmax = 3,
		ymin = 0, ymax = 4,
		xtick = {0}, ytick = {0},
		extra x ticks = {1, 2.1},
		extra x tick labels = {\(T\), \(2T\)},
		extra y ticks = {3},
		extra y tick labels = {\SI{3}{\volt}},
	]

	\addplot[color=blue, mark=none] coordinates {
		(0, 3) (1, 3) (1, 0) (1.1, 0) (1.1, 3) (2.1, 3) (2.1, 0)
		(2.2, 0) (2.2, 3) (3, 3)
	};
	\end{axis}
\end{tikzpicture}
\end{center}

\subsection{}

The generalized expression for \(V_{DD}\) can be represented with the differential equation
\begin{align}
	C \diff{t} V_{DD} + i(t) - \frac{V_s - V_{DD}}{R} &= 0 \\
	\diff{t} V_{DD} + \frac{V_{DD}}{RC} = \frac{V_s}{RC} - \frac{i(t)}{C}
\end{align}
with \(i(0) = 0\).

\begin{center}
\begin{tikzpicture}
	\begin{axis}[
		title = {\(C = \SI{1}{\pico\farad}\)},
		xlabel = {\(t [\si{\nano\second}]\)},
		ylabel = {\(V_{out1}(t) [\si{\volt}]\)},
		xmin = 0, xmax = 3,
		ymin = 0, ymax = 4,
		xtick = {0}, ytick = {0},
		extra x ticks = {1, 2.1},
		extra x tick labels = {\(T\), \(2T\)},
		extra y ticks = {3},
		extra y tick labels = {\SI{3}{\volt}},
	]
	
		\addplot[color=blue, mark=none] coordinates {
			(0, 3) (1, 3) (1, 0) (1.1, 0) (1.1, 3) (2.1, 3) (2.1, 0)
			(2.2, 0) (2.2, 3) (3, 3)
		};
	\end{axis}
\end{tikzpicture}
\begin{tikzpicture}
	\begin{axis}[
		title = {\(C = \SI{1}{\nano\farad}\)},
		xlabel = {\(t [\si{\nano\second}]\)},
		ylabel = {\(V_{out1}(t) [\si{\volt}]\)},
		xmin = 0, xmax = 3,
		ymin = 0, ymax = 4,
		xtick = {0}, ytick = {0},
		extra tick style = {grid=major},
		extra x ticks = {1, 2.1},
		extra x tick labels = {\(T\), \(2T\)},
		extra y ticks = {3},
		extra y tick labels = {\SI{3}{\volt}},
	]
		
		\addplot[domain=0:1, color=blue] {3*(1 - exp(-10*x))};
		\addplot[domain=1:1.1, color=blue] {3*exp(-100*(x - 1))};
		\addplot[domain=1.1:2.1, color=blue] {3*(1 - exp(-10*(x - 1.1)))};
		\addplot[domain=2.1:2.2, color=blue] {3*exp(-100*(x - 2.1))};
		\addplot[domain=2.2:3, color=blue] {3*(1 - exp(-10*(x - 2.2)))};
	\end{axis}
\end{tikzpicture}
\begin{tikzpicture}
	\begin{axis}[
		title = {\(C = \SI{1}{\micro\farad}\)},
		xlabel = {\(t [\si{\nano\second}]\)},
		ylabel = {\(V_{out1}(t) [\si{\volt}]\)},
		xmin = 0, xmax = 3,
		ymin = 0, ymax = 4,
		xtick = {0}, ytick = {0},
		extra tick style = {grid=major},
		extra x ticks = {1, 2.1},
		extra x tick labels = {\(T\), \(2T\)},
		extra y ticks = {3},
		extra y tick labels = {\SI{3}{\volt}},
	]
	
		\addplot[domain=0:1, color=blue] {3};
		\addplot[domain=1:1.1, color=blue] {(-0.5/0.1)*(x - 1) + 3};
		\addplot[domain=1.1:2.1, color=blue] {2.5};
		\addplot[domain=2.1:2.2, color=blue] {(-0.5/0.1)*(x - 2.1) + 2.5};
		\addplot[domain=2.2:3, color=blue] {2};
	\end{axis}
\end{tikzpicture}
\end{center}

\subsection{}

It is better to have a higher \(C\) than a higher \(R\).
This is because a higher capacitance means that the capacitor will take longer to discharge completely, resisting larger changes in voltage, possibly outlasting \(t_p\).

\section{Simple Scalar DEs Driven by an Input}

\subsection{}

\begin{proof}
We will prove that \(\delta(t) = x_g(t) - y(t) = 0\), thuss proving uniqueness.
First proving the initial conditions cancel,
\begin{equation}
	\delta(t) = x_g(0) - y(0) = x_0 - x_0 = 0
\end{equation}
Next, we prove uniqueness of the derivative:
\begin{align}
	\diff{t} \delta(t) &= \diff{t} (x_g - y) \\
	&= \lambda x_g + \cancel{u(t)} - \lambda y - \cancel{u(t)} \\
	&= \lambda (x_g - y)
\end{align}
\end{proof}

\subsection{}

\begin{proof}
\begin{align}
	x_c(t) &= x_0 e^{\lambda t} + \int_0^t u(\tau) e^{\lambda (t - \tau)} \, d\tau \\
	x_c(0) &= x_0 \cancelto{1}{e^{\lambda \cdot 0}} + \cancel{\int_0^0 u(\tau) e^{\lambda (t - \tau)} \, d\tau} = x_0 \\
	\diff{t} x_c(t) &= \lambda x_0 e^{\lambda t} + \underbrace{\int_0^t \frac{\partial}{\partial t} u(\tau) e^{\lambda (t - \tau)} \, d\tau}_{\text{Leibniz integral rule}} + \underbrace{u(t) \cancelto{1}{e^{\lambda (t - t)}}}_{\text{FTC}} \\
	&= \lambda x_c(t) + \lambda \int_0^t u(\tau) e^{\lambda (t - \tau)} \, d\tau + u(t) \\
	&= \lambda x_c(t) + u(t)
\end{align}
\end{proof}

\subsection{}

Plugging in,
\begin{align}
	x_c(t) &= x_0 e^{\lambda t} + \int_0^t e^{s \tau} e^{\lambda (t - \tau)} \, d\tau = x_0 e^{\lambda t} + \int_0^t e^{(s - \lambda) \tau + \lambda t} \, d\tau \\
	&= x_0 e^{\lambda t} + \left.\frac{1}{s - \lambda} e^{(s - \lambda) \tau + \lambda t} \right|_0^t \\
	&= x_0 e^{\lambda t} + \left(\frac{1}{s - \lambda} e^{st - \lambda t + \lambda t} \right) - \left(\frac{1}{s - \lambda} e^{\lambda t} \right) \\
	&= x_0 e^{\lambda t} + \frac{1}{s - \lambda} (e^{st} - e^{\lambda t})
\end{align}

\subsection{}

Plugging in,
\begin{align}
	x_c(t) &= x_0 e^{\lambda t} + \int_0^t e^{\lambda \tau} e^{\lambda (t - \tau)} \, d\tau = x_0 e^{\lambda t} + \int_0^t e^{\lambda t} \, d\tau \\
	&= x_0 e^{\lambda t} + t e^{\lambda t} \, d\tau
\end{align}

\section{Op-Amp Stability}

\subsection{}

\begin{center}
\begin{circuitikz}\draw
	(-2, 2) to[short] (-1, 2) node[ocirc]{}
	(-2, 0) to[short] (-1, 0) node[ocirc]{\ \(V_{in}\)}
	(0, 0) node[ground]{} to[cvsource, v=\(G \Delta V\), invert] (0, 2) to[R=\(R_{out}\)] (2, 2) to[C=\(C_{out}\)] (2, 0) node[ground]{}
	(2, 2) to[short] (4, 2) node[ocirc]{\ \(V_{out}\)}
	(3, 2) to[short] (3, 3) to[short] (-2, 3) to[short] (-2, 2)
;\end{circuitikz}
\end{center}

\subsection{}

The KCL equation for \(V_{out}\) is
\begin{align}
	C_{out} \diff{t} V_{out} - \frac{G (V_{in} - V_{out}) - V_{out}}{R_{out}} &= 0 \\
	\diff{t} V_{out} = \frac{G V_{in} - (G + 1) V_{out}}{R_{out} C_{out}} &= \frac{G + 1}{R_{out} C_{out}} \left(\frac{G}{G + 1} V_{in} - V_{out}\right) \\
	\int \frac{1}{\frac{G}{G + 1} V_{in} - V_{out}} \, dV_{out} &= \int \frac{G + 1}{R_{out} C_{out}} \, dt \\
	-\ln\left|\frac{G}{G + 1} V_{in} - V_{out}\right| &= \frac{G + 1}{R_{out} C_{out}} t + K \\
	V_{out}(t) &= \frac{G}{G + 1} V_{in} - V_0 \exp\left(-\frac{G + 1}{R_{out} C_{out}} t\right) \\
	\lim_{t \to \infty} V_{out}(t) &= \frac{G}{G + 1} V_{in}
\end{align}

\subsection{}

Using similar techniques,
\begin{align}
	C_{out} \diff{t} V_{out} - \frac{G (V_{out} - V_{in}) - V_{out}}{R_{out}} &= 0 \\
	\diff{t} V_{out} = \frac{(G - 1) V_{out} - G V_{in}}{R_{out} C_{out}} &= \frac{G - 1}{R_{out} C_{out}} \left(V_{out} - \frac{G}{G - 1} V_{in}\right) \\
	\int \frac{1}{V_{out} - \frac{G}{G - 1} V_{in}} \, dV_{out} &= \int \frac{G - 1}{R_{out} C_{out}} \, dt \\
	\ln\left|V_{out} - \frac{G}{G - 1} V_{in}\right| &= \frac{G - 1}{R_{out} C_{out}} t + K \\
	V_{out}(t) &= \frac{G}{G - 1} V_{in} + V_0 \exp\left(\frac{G - 1}{R_{out} C_{out}} t\right) \\
	\lim_{t \to \infty} V_{out}(t) &= \pm \infty
\end{align}
Depending on the initial condition, the op-amp will rail either to the positive or negative.
In fact, for \(V_{in} > 0\), the op-amp will rail negative due to the fact that when we solve for the initial condition, \(V_0 = -\frac{G}{G - 1} V_{in} < 0\).

\subsection{}

Given the \(V_{out}(t)\) from \textbf{4.b},
\begin{equation}
	\lim_{G \to \infty} V_{out}(t) = V_{in} - \cancel{V_0 \exp\left(\lim_{G \to \infty} -\frac{G + 1}{R_{out} C_{out}} t\right)} = V_{in}
\end{equation}

\stepcounter{section}

\section{Homework Process and Study Group}

\begin{enumerate}
	\item I referred to my lecture notes.
	\item I did this homework by myself.
	\item I worked on this homework in one sitting, and revised it the day after.
	\item 4 hours.
\end{enumerate}

\newpage

%\includepdf[pages=-]{prob*.pdf}

\end{document}
