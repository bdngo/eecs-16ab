\documentclass[]{article}
\usepackage{amsmath, amsfonts, amssymb, amsthm}
\usepackage{cancel}
\usepackage{siunitx}
\usepackage[american]{circuitikz}
\usepackage{pgfplots}
\usepackage{bm}
\usepackage{graphicx}
\usepackage{pdfpages}

\renewcommand{\thesection}{\arabic{section}}
\renewcommand{\thesubsection}{\thesection.\alph{subsection}}
\renewcommand{\thesubsubsection}{\thesubsection.\roman{subsubsection}}

\newtheorem{genthm}{Theorem}

\newcommand{\unit}[1]{\bm{\hat{#1}}}
\newcommand{\iprod}[2]{\left\langle #1, #2 \right\rangle}
\newcommand{\tpose}[1]{\left[#1\right]^{\! \top} \!\!}

\title{EECS 16B HW01}
\author{Bryan Ngo}
\date{2020-01-25}

\begin{document}

\maketitle

\section{Existence and Uniqueness of Solutions to Differential Equations}

\subsection{}

\begin{align}
	\frac{d}{dt} x_d(t) &= \alpha x_0 e^{\alpha t} = \alpha x_d(t) \\
	x_d(0) &= x_0 e^0 = x_0
\end{align}

\subsection{}

\begin{equation}
	z(0) = \frac{y(0)}{x(0)} = \frac{x_0}{x_0} = 1
\end{equation}

\subsection{}

\begin{align}
	\frac{d}{dt} z(t) &= \frac{y'(t) x(t) - x'(t) y(t)}{x(t)^2} \\
	&= \frac{\cancel{\alpha y(t) x(t)} - \cancel{\alpha x(t) y(t)}}{x(t)^2} = 0
\end{align}
Since \(\frac{d}{dt} z(t) = 0\), this implies that \(z(t)\) is a constant.
Specifically, \(z(t) = 1\) due to the ratio of their initial conditions as we have shown above.

\subsection{}

Applying the transformation \(t = t_0 - \tau\),
\begin{equation}
	\frac{d}{d\tau} \tilde{x}(\tau) = \frac{d}{d\tau} x(t_0 - \tau) = -\alpha x(t_0 - \tau) = -\alpha \tilde{x}(\tau)
\end{equation}

\subsection{}

Using the equation from \textbf{1.d},
\begin{align}
	\frac{d}{d\tau} \tilde{y}(\tau) &= -\alpha \tilde{y}(\tau) \\
	\Rightarrow \tilde{y}(\tau) &= x_0 e^{-\alpha (t_0 - \tau)}
\end{align}

% TODO: add explanation/fix?

\subsection{}

\begin{equation}
	\tilde{y}(0) = x_0 e^{-\alpha (t_0)} \neq x_0
\end{equation}
By contradiction, we establish uniqueness for all \(x_0 \in \mathbb{R}\).

\subsection{}

It is important to establish unique solutions to differential equations so that we can derive important information.
If there were non-unique solutions, then multiple equations could describe behavior.

\section{Digital-Analog Converter}

\subsection{}

\begin{center}
\begin{circuitikz}\draw
	(0, 0) node[ground]{} to[short] (0, 2) to[R=\(2R\)] (0, 4) to[short] (2, 4) to[R=\(R\)] (4, 4) to[R=\(R\)] (6, 4) to[short] (7, 4) node[ocirc]{\ \(V_{out}\)}
	(2, 0) node[ground]{} to[short] (2, 2) to[R=\(2R\)] (2, 4)
	(4, 0) node[ground]{} to[short] (4, 2) to[R=\(2R\)] (4, 4)
	(6, 0) node[ground]{} to[V=\(V_2\), invert] (6, 2) to[R=\(2R\)] (6, 4)
;\end{circuitikz}
\end{center}

The left nest of resistors ends up having an equivalent resistance of \(2R\).
We can treat the resulting circuit as a voltage divider, with voltage
\begin{equation}
	V_{out} = V_2 \frac{2\cancel{R}}{4\cancel{R}} = \frac{1}{2} V_{DD}
\end{equation}

\subsection{}

\begin{center}
\begin{circuitikz}\draw
	(0, 0) node[ground]{} to[short] (0, 2) to[R=\(2R\)] (0, 4) to[short] (2, 4) to[R=\(R\)] (4, 4) to[R=\(R\)] (6, 4) to[short] (7, 4) node[ocirc]{\ \(V_{out}\)}
	(2, 0) node[ground]{} to[short] (2, 2) to[R=\(2R\)] (2, 4)
	(4, 0) node[ground]{} to[V=\(V_1\), invert] (4, 2) to[R=\(2R\)] (4, 4)
	(6, 0) node[ground]{} to[short] (6, 2) to[R=\(2R\)] (6, 4)
;\end{circuitikz}
\end{center}
Simplifying and using NVA, we construct the matrix equation
\begin{align}
	&\left[\begin{array}{@{}cc|c@{}}
	2 & -1 & \frac{V_1}{2} \\
	-1 & \frac{3}{2} & 0
	\end{array}\right] \\
	&\Longrightarrow
	\left[\begin{array}{@{}cc|c@{}}
	1 & 0 & \frac{3 V_1}{8} \\
	0 & 1 & \frac{V_1}{4}
	\end{array}\right]
\end{align}
where \(V_{out} = \frac{1}{4}V_{DD}\).

\subsection{}

\begin{center}
\begin{circuitikz}\draw
	(0, 0) node[ground]{} to[short] (0, 2) to[R=\(2R\)] (0, 4) to[short] (2, 4) to[R=\(R\)] (4, 4) to[R=\(R\)] (6, 4) to[short] (7, 4) node[ocirc]{\ \(V_{out}\)}
	(2, 0) node[ground]{} to[V=\(V_0\), invert] (2, 2) to[R=\(2R\)] (2, 4)
	(4, 0) node[ground]{} to[short] (4, 2) to[R=\(2R\)] (4, 4)
	(6, 0) node[ground]{} to[short] (6, 2) to[R=\(2R\)] (6, 4)
;\end{circuitikz}
\end{center}
Simplifying and using NVA, we construct the matrix equation
\begin{align}
	&\left[\begin{array}{@{}ccc|c@{}}
	2 & -1 & 0 & \frac{V_1}{2} \\
	-1 & \frac{5}{2} & -1 & 0 \\
	0 & -1 & \frac{3}{2} & 0
	\end{array}\right] \\
	&\Longrightarrow
	\left[\begin{array}{@{}ccc|c@{}}
	1 & 0 & 0 & \frac{11 V_1}{32} \\
	0 & 1 & 0 & \frac{3 V_1}{16} \\
	0 & 0 & 1 & \frac{V_1}{8}
	\end{array}\right]
\end{align}
where \(V_{out} = \frac{1}{8}V_{DD}\).

\subsection{}

By the principle of superposition, the DAC's output voltage when all three bits are on is simply the addition of each individual bit's voltage:
\begin{equation}
	V_{out} = \frac{7}{8} V_{DD}
\end{equation}

\subsection{}

\begin{equation}
	V_{out} = V_{DD} \left(\frac{1}{2} b_2 + \frac{1}{4} b_1 + \frac{1}{8} b_0\right)
\end{equation}

\subsection{}

The DAC computes an analog voltage by turning the value of each bit into a weighted sum that is then applied to the supply voltage.
Each bit corresponds to a power of two, and is thus given an appropriate weight.

\section{Transistor Switch Model}

\subsection{}

We can model the node \(V_{out1}\) as
\begin{center}
\begin{circuitikz}\draw
	(0, 0) node[ground]{} to[R=\(R_n\)] (0, 2) to[short] (2, 2) node[circ]{\ \(V_{out1}\)} to[C=\(C_n\)] (2, 0) node[ground]{}
	(2, 2) to[C=\(C_p\)] (2, 4) node[vcc]{\(V_{DD}\)}
;\end{circuitikz}
\end{center}
Writing out the node equation for \(V_{out1}\) yields
\begin{equation}
	\frac{V_{out1}}{R} + C_n \frac{d}{dt} V_{out1} - C_p \frac{d}{dt} (V_{DD} - V_{out1}) = 0
\end{equation}

\subsection{}

\begin{align}
	\frac{V_{out1}}{R} + C_n \frac{d}{dt} V_{out1} - \cancel{C_p \frac{d}{dt} V_{DD}} + C_p \frac{d}{dt} V_{out1} &= 0 \\
	\frac{d}{dt} V_{out1} &= -\frac{V_{out1}}{\underbrace{R (C_p + C_n)}_{\tau}} \\
	\int \frac{1}{V_{out1}} \, dV_{out1} &= \int -\frac{1}{\tau} \, dt \\
	\ln|V_{out1}| &= -\frac{1}{\tau} t + C \\
	V_{out1} &= V_0 e^{-\frac{1}{\tau} t}
\end{align}
Plugging in our initial condition \(V_{out1}(0) = V_{DD}\),
\begin{equation}
	V_{out1}(0) = V_0 = V_{DD}
\end{equation}
So our solution is
\begin{equation}
	V_{out1}(t) = V_{DD} e^{-\frac{1}{\tau} t}
\end{equation}

\subsection{}

\begin{center}
\begin{tikzpicture}
	\begin{axis}[
		xlabel = {\(t\)},
		ylabel = {\(V_{out1}(t)\)},
		xmin = 0, xmax = 5,
		ymin = 0, ymax = 1,
		xtick = {0},	ytick = {0},
		extra tick style = {grid=major},
		extra x ticks = {ln(3)},
		extra x tick labels = {\(\tau \ln(3)\)},
		extra y ticks = {1, 1/3},
		extra y tick labels = {\(V_{DD}\), \(\frac{1}{3} V_{DD}\)},
	]
		\addplot[
			domain = 0:5,
			samples = 100,
			color = blue
		]
		{exp(-x)};
		\addlegendentry{\(V_{DD} e^{-\frac{1}{\tau} t}\)}
		
		\addplot[
		domain = 0:5,
		samples = 100,
		color = red
		]
		{1-x};
		\addlegendentry{\(V_{DD} \left(1 - \frac{1}{\tau} t\right)\)}
	\end{axis}
\end{tikzpicture}
\end{center}

\subsection{}

\begin{center}
\begin{circuitikz}\draw
	(0, 4) node[vcc]{\(V_{DD}\)} to[R=\(R_p\)] (0, 2) to[short] (2, 2) node[circ]{\ \(V_{out1}\)} to[C=\(C_p\)] (2, 4) node[vcc]{\(V_{DD}\)}
	(2, 2) to[C=\(C_n\)] (2, 0) node[ground]{}
;\end{circuitikz}
\end{center}
The node equation is
\begin{equation}
	C_n \frac{d}{dt} V_{out1} - C_p \frac{d}{dt} (\cancel{V_{DD}} - V_{out1}) - \frac{V_{DD} - V_{out1}}{R_p} = 0
\end{equation}
Solving,
\begin{align}
	C_n \frac{d}{dt} V_{out1} + C_p \frac{d}{dt} V_{out1} - \frac{V_{DD} - V_{out1}}{R_p} &= 0 \\
	\frac{d}{dt} V_{out1} &= \frac{V_{DD} - V_{out1}}{\underbrace{R_p (C_p + C_n)}_{\tau}} \\
	\int \frac{1}{V_{DD} - V_{out1}} \, dV_{out1} &= \int \frac{1}{\tau} \, d\tau \\
	-\ln|V_{DD} - V_{out1}| &= \frac{1}{\tau} + C \\
	V_{DD} - V_{out1} &= V_0 e^{-\frac{1}{\tau} t} \\  
	V_{out1}(t) &= V_{DD} - V_0 e^{-\frac{1}{\tau} t}
\end{align}
Plugging in the initial condition \(V_{out1}(0) = 0\),
\begin{align}
	0 &= V_{DD} - V_0 \\
	V_0 &= V_{DD}
\end{align}
Yielding the final equation
\begin{equation}
	V_{out1}(t) = V_{DD} (1 - e^{-\frac{1}{\tau} t})
\end{equation}

\begin{center}
\begin{tikzpicture}
	\begin{axis}[
		xlabel = {\(t\)},
		ylabel = {\(V_{out1}(t)\)},
		xmin = 0, xmax = 5,
		ymin = 0, ymax = 1,
		xtick = {0}, ytick = {0},
		extra tick style = {grid=major},
		extra y ticks = {1},
		extra y tick labels = {\(V_{DD}\)},
	]
		\addplot[
		domain = 0:5,
		samples = 100,
		color = blue
		]
		{1 - exp(-x)};
		\addlegendentry{\(V_{DD} (1 - e^{-\frac{1}{\tau} t})\)}
		
		\addplot[
		domain = 0:5,
		samples = 100,
		color = red
		]
		{x};
		\addlegendentry{\(\frac{V_{DD}}{\tau}t\)}
	\end{axis}
\end{tikzpicture}
\end{center}

\subsection{}

Since the circuit goes through a voltage cycle of \(V_{DD}\), then the total charge is \(V_{DD} (C_p + C_n)\) by definition of a capacitor.
Plugging in, this gives us \SI{2}{\femto\coulomb}.

\section{RC Circuit}

\subsection{}

\begin{align}
	I(0) &= \frac{V_s}{R} \\
	\lim_{t \to \infty} I(t) &= 0
\end{align}

\subsection{}

\begin{align}
	V_s - V_R(t) - V_C(t) &= 0 \\
	V_s - I_C(t)R - \frac{1}{C} Q_C(t) &= 0 \\
	R\frac{d}{dt} I_C(t) + \frac{1}{C} I_C(t) &= 0
\end{align}

\subsection{}

The eigenvalue \(\lambda = -\frac{1}{RC}\).

\subsection{}

\begin{align}
	\frac{d}{dt} I_C(t) &= -\frac{1}{RC} I_C(t) \\
	\int \frac{1}{I_C} \, dI_C &= \int -\frac{1}{RC} \, dt \\
	\ln|I_C| &= -\frac{1}{RC} t + K \\
	I_C(t) &= I_0 e^{-\frac{1}{RC} t}
\end{align}
Plugging in the initial values,
\begin{align}
	I_C(0) &= I_0 = \frac{V_s}{R} \\
	I_C(t) &= \frac{V_s}{R} e^{-\frac{1}{RC} t}
\end{align}

\subsection{}

Solving for \(0.05 \frac{V_s}{R}\),
\begin{align}
	0.05 &= e^{-\frac{1}{RC} t} \\
	\ln(0.05) &= -\frac{1}{RC} t \\
	t &= -RC \cdot \ln(0.05) \rightsquigarrow \SI{2.99}{\milli\second}
\end{align}

\subsection{}

\begin{enumerate}
	\item Make \(R\) smaller.
	\item Make \(C\) smaller.
\end{enumerate}

\subsection{}

\begin{center}
\begin{tikzpicture}
	\begin{axis}[
		xlabel = {\(t\) [\si{\milli\second}]},
		ylabel = {\(I_C(t)\) [\si{\ampere}]},
		xmin = 0, xmax = 0.005, xtick = \empty,
		ymin = 0, ymax = 0.05, ytick = \empty,
		extra x ticks = {0.001, 0.002, 0.003},
		extra x tick labels = {\(\tau\), \(2\tau\), \(3\tau\)},
		extra y ticks = {0.05/exp(1), 0.05/exp(2), 0.05/exp(3)},
		extra y tick labels = {\(I_C(\tau)\), \(I_C(2\tau)\), \(I_C(3\tau)\)},
		extra tick style = {grid=major}
	]
		\addplot[
		domain = 0:0.005,
		samples = 100,
		color = blue
		]
		{0.05*exp(-1000*x)};
		\addlegendentry{\(\frac{V_s}{R} e^{-\frac{1}{RC} t}\)}
	\end{axis}
\end{tikzpicture}
\end{center}

\stepcounter{section}

\section{Homework Process and Study Group}

\renewcommand{\theenumi}{\alph{enumi}}
\begin{enumerate}
	\item I used Note 3 from 28 January.
	\item I worked on this homework by myself.
	\item I worked on this homework in one sitting.
	\item I spend 4 hours on this homework.
\end{enumerate}

\newpage

%\includepdf[pages=-]{prob*.pdf}

\end{document}
