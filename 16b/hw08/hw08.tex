\documentclass[]{article}
\usepackage{amsmath, amsfonts, amssymb, amsthm}
\usepackage{cancel}
\usepackage{pgfplots}
\usepackage{siunitx}
\usepackage[american]{circuitikz}
\usepackage{bm}
\usepackage{graphicx}
\usepackage{fullpage}
\usepackage{pdfpages}

\renewcommand{\thesection}{\arabic{section}}
\renewcommand{\thesubsection}{\thesection.\alph{subsection}}
\renewcommand{\thesubsubsection}{\thesubsection.\roman{subsubsection}}
\renewcommand{\geq}{\geqslant}
\renewcommand{\leq}{\leqslant}

\newtheorem{genthm}{Theorem}

\newcommand{\unit}[1]{\bm{\hat{#1}}}
\newcommand{\iprod}[2]{\left\langle #1, #2 \right\rangle}
\newcommand{\tpose}[1]{\left[#1\right]^{\! \top} \!\!}
\newcommand{\diff}[1]{\frac{d}{d #1}}

\title{EECS 16B HW08}
\author{Bryan Ngo}
\date{2020-03-27}

\begin{document}

\maketitle

\section{SVD}

\begin{equation}
	\bm{A} = \begin{bmatrix}
	-1 & 1 & 5 \\
	3 & 1 & -1 \\
	2 & -1 & 4
	\end{bmatrix}
\end{equation}

\subsection{}

\begin{align}
	\tpose{\bm{A}} \bm{A} &= \begin{bmatrix}
	-1 & 3 & 2 \\
	1 & 1 & -1 \\
	5 & -1 & 4
	\end{bmatrix}
	\begin{bmatrix}
	-1 & 1 & 5 \\
	3 & 1 & -1 \\
	2 & -1 & 4
	\end{bmatrix}
	=
	\begin{bmatrix}
	14 & 0 & 0 \\
	0 & 3 & 0 \\
	0 & 0 & 42
	\end{bmatrix} \\
	\bm{A} \tpose{\bm{A}} &= \begin{bmatrix}
	-1 & 1 & 5 \\
	3 & 1 & -1 \\
	2 & -1 & 4
	\end{bmatrix}
	\begin{bmatrix}
	-1 & 3 & 2 \\
	1 & 1 & -1 \\
	5 & -1 & 4
	\end{bmatrix}
	=
	\begin{bmatrix}
	27 & -7 & 17 \\
	-7 & 11 & 1 \\
	17 & 1 & 21
	\end{bmatrix}
\end{align}

\subsection{}

Using \(\tpose{\bm{A}} \bm{A}\),
\begin{align}
	\lambda &= \{42 \geq 14 \geq 3\} \\
	\bm{v} &= \{\unit{k}, \unit{\imath}, \unit{\jmath}\}
\end{align}
Thus, our \(\bm{u}_i\) is
\begin{align}
	\bm{u}_1 &= \frac{1}{\sqrt{42}} \bm{A} \bm{v}_1 = \frac{1}{\sqrt{42}} \begin{bmatrix}
	5 \\
	-1 \\
	4
	\end{bmatrix} \\
	\bm{u}_2 &= \frac{1}{\sqrt{14}} \begin{bmatrix}
	-1 \\
	3 \\
	2
	\end{bmatrix} \\
	\bm{u}_3 &= \frac{1}{\sqrt{3}} \begin{bmatrix}
	1 \\
	1 \\
	-1
	\end{bmatrix}
\end{align}
So our SVD is
\begin{equation}
	\bm{A} = \frac{1}{\sqrt{42}} \begin{bmatrix}
	5 \\
	-1 \\
	4
	\end{bmatrix} \begin{bmatrix}
	0 & 0 & 1
	\end{bmatrix}
	+
	\frac{1}{\sqrt{14}} \begin{bmatrix}
	-1 \\
	3 \\
	2
	\end{bmatrix} \begin{bmatrix}
	1 & 0 & 0
	\end{bmatrix}
	+
	\frac{1}{\sqrt{3}} \begin{bmatrix}
	1 \\
	1 \\
	-1
	\end{bmatrix} \begin{bmatrix}
	0 & 1 & 0
	\end{bmatrix}
\end{equation}

\section{Rank 1 Decomposition}

\subsection{}

\(\bm{C}_1\) has rank 2, so we need 2 rank 1 matrices to represent it.
\begin{equation}
	\bm{C}_1 = \frac{1}{\sqrt{8}} \begin{bmatrix}
	1 \\
	-1 \\
	1 \\
	-1 \\
	1 \\
	-1 \\
	1 \\
	-1
	\end{bmatrix} \begin{bmatrix}
	1 & 0 & 0 & 0 & 0 & 0 & 0 & 0
	\end{bmatrix}
	+
	\frac{1}{\sqrt{8}} \begin{bmatrix}
	-1 \\
	1 \\
	-1 \\
	1 \\
	-1 \\
	1 \\
	-1 \\
	1
	\end{bmatrix} \begin{bmatrix}
	0 & 1 & 0 & 0 & 0 & 0 & 0 & 0
	\end{bmatrix}
\end{equation}

\subsection{}

\begin{equation}
	\bm{C}_2 = \frac{1}{2} \begin{bmatrix}
	1 \\
	0 \\
	1 \\
	0 \\
	1 \\
	0 \\
	1 \\
	0
	\end{bmatrix} \begin{bmatrix}
	1 & 0 & 1 & 0 & 1 & 0 & 1 & 0
	\end{bmatrix}
	+
	\frac{1}{2} \begin{bmatrix}
	0 \\
	1 \\
	0 \\
	1 \\
	0 \\
	1 \\
	0 \\
	1
	\end{bmatrix} \begin{bmatrix}
	0 & 1 & 0 & 1 & 0 & 1 & 0 & 1
	\end{bmatrix}
\end{equation}

\subsection{}

\subsubsection{}

\begin{equation}
	\bm{F}_{\text{CH}} = \frac{1}{\sqrt{3}} \begin{bmatrix}
	0 \\
	1 \\
	1 \\
	1 \\
	0
	\end{bmatrix} \begin{bmatrix}
	0 & 1 & 1 & 1 & 0
	\end{bmatrix}
	+
	\frac{1}{\sqrt{2}} \begin{bmatrix}
	0 \\
	1 \\
	0 \\
	1 \\
	0
	\end{bmatrix} \begin{bmatrix}
	0 & 1 & 0 & 1 & 0
	\end{bmatrix}
\end{equation}

\subsubsection{}

\begin{equation}
	\bm{F}_{\text{CH}} = \frac{1}{\sqrt{3}} \begin{bmatrix}
	0 \\
	1 \\
	1 \\
	1 \\
	0
	\end{bmatrix} \begin{bmatrix}
	0 & 0 & 1 & 0 & 0
	\end{bmatrix}
	+
	\begin{bmatrix}
	0 \\
	0 \\
	1 \\
	0 \\
	0
	\end{bmatrix} \begin{bmatrix}
	0 & 1 & 0 & 1 & 0
	\end{bmatrix}
\end{equation}

\section{Open-Loop Control of SIXT33N}

\begin{gather}
	u[t] = \frac{v^\ast + \beta}{\theta} \\
	v_L[t] = d_L[t + 1] - d_L[t] = \theta_L u_L[t] - \beta_L \\
	v_R[t] = d_R[t + 1] - d_R[t] = \theta_R u_R[t] - \beta_R
\end{gather}

\subsection{}

Inductively finding \(d_L[1], d_R[1]\),
\begin{align}
	d_L[1] &= v_L[0] + \cancelto{0}{d_L[0]} = \theta_L u_L[0] - \beta_L = 255\theta_L - \beta_L \\
	d_L[t + 1] &= v_L[t] + d_L[t] = \theta_L u_L[t] - \beta_L + t(255\theta_L - \beta_L) \\
	d_L[t_p] &= t_p(255\theta_L - \beta_L) \\
	d_R[t_p] &= t_p(255\theta_R - \beta_R)
\end{align}

\subsection{}

\begin{equation}
	\delta[t_p] = t_p(255\theta_L - 255\theta_R - \beta_L + \beta_R)
\end{equation}
If \(\theta_L = \theta_R\) and \(\beta_L = \beta_R\), \(\delta[t_p] = 0\), so the car does not turn after the pulse.
Similarly, if \(\theta_L \neq \theta_R\) and \(\beta_L \neq \beta_R\), \(\delta[t_p] \neq 0\) except in a few degenerate cases, so the car will generally be turning after the pulse.

\subsection{}

By induction,
\begin{align}
	\delta[t_p + 1] &= d_L[t_p + 1] - d_R[t_p + 1] \\
	&= \left(\theta_L \frac{v^\ast + \beta_L}{\theta_L} - \beta_L\right) - \left(\theta_R \frac{v^\ast + \beta_R}{\theta_R} - \beta_R\right) + \delta_0 = \delta_0 \\
	\Rightarrow \lim_{t \to \infty} \delta[t] &= \delta_0
\end{align}
The car does not seem to deviate from \(\delta_0\).

\subsection{}

\begin{align}
	\delta[t_p + 1] &= \left((\theta_L + \Delta \theta_L) \frac{v^\ast + \beta_L}{\theta_L} - (\beta_L + \Delta \beta_L)\right) - \left((\theta_R + \Delta \theta_R) \frac{v^\ast + \beta_R}{\theta_R} - (\beta_R + \Delta \beta_R)\right) + \delta_0 \\
	&= \left(\frac{\Delta \theta_L (v^\ast + \beta_L)}{\theta_L} - \Delta \beta_L\right) - \left(\frac{\Delta \theta_R (v^\ast + \beta_R)}{\theta_R} - \Delta \beta_R\right) + \delta_0 \\
	\Rightarrow \delta[t]|_{t > t_p} &= t \cdot \delta[t_p + 1]
\end{align}
The car seems to deviate from \(\delta_0\), assuming nonzero deviance factors.

\section{Closed-Loop Control of SIXT33N}

\begin{align}
	d_L[k + 1] - d_L[k] &= v^\ast - k_L\delta[k] \\
	d_R[k + 1] - d_R[k] &= v^\ast - k_R\delta[k] \\
	u_L[k] &= \frac{v^\ast + \beta_L}{\theta_L} - k_L \frac{\delta[k]}{\theta_L} \\
	u_R[k] &= \frac{v^\ast + \beta_R}{\theta_R} + k_R \frac{\delta[k]}{\theta_R}
\end{align}

\subsection{}

\begin{align}
	\delta[k + 1] &= d_L[k + 1] - d_R[k + 1] \\
	&= \cancel{v^\ast} - k_L\delta[k] + d_L[k] - \cancel{v^\ast} - k_R\delta[k] - d_R[k] \\
	&= (1 - k_L - k_R) \delta[k]
\end{align}

\subsection{}

The eigenvalue is \(\lambda = 1 - k_L - k_R\).
\(\lambda \in (-1, 0] \cup [0, 1)\) is functionally identical to \(\lambda \in (-1, 1)\) because \(0 \in (-1, 1)\).
Finding the values of \(k_L, k_R\),
\begin{align}
	1 - k_L - k_R &\in (-1, 1) \\
	-k_L - k_R &\in (-2, 0) \\
	k_L + k_R &\in (0, -2)
\end{align}

\subsection{}

Substituting \(u_L, u_R\) into their respective distance equations,
\begin{align}
	\delta[k + 1] &= \left((\theta_L + \Delta \theta_L) \left(\frac{v^\ast + \beta_L - k_L \delta[k]}{\theta_L}\right) - (\beta_L + \Delta \beta_L)\right) - \left((\theta_R + \Delta \theta_R) \left(\frac{v^\ast + \beta_R + k_R \delta[k]}{\theta_R}\right) - (\beta_R + \Delta \beta_R)\right) + \delta[k] \\
	&= \left(\Delta \theta_L \left(\frac{v^\ast + \beta_L - k_L \delta[k]}{\theta_L}\right) - k_L \delta[k] - \Delta \beta_L\right) - \left(\Delta \theta_R \left(\frac{v^\ast + \beta_R - k_R \delta[k]}{\theta_R}\right) + k_R \delta[k] - \Delta \beta_R\right) + \delta[k] \\
	&= \underbrace{\left(\Delta \theta_L \left(\frac{v^\ast + \beta_L}{\theta_L}\right) - \Delta \beta_L\right) - \left(\Delta \theta_R \left(\frac{v^\ast + \beta_R}{\theta_R}\right) - \Delta \beta_R\right)}_{c} + \underbrace{\left(1 - k_L - k_R + \frac{\Delta \theta_L k_L}{\theta_L} + \frac{\Delta \theta_R k_R}{\theta_R}\right)}_{\lambda} \delta[k]
\end{align}
This is better than open-loop control because we reduce the difference by an exponential amount rather than having it explode to infinity.

\stepcounter{section}

\section{Homework Process and Study Group}

\begin{enumerate}
	\item I used Lecture 9B notes.
	\item I worked on this homework by myself.
	\item I worked on this homework in one sitting,
	\item Questions 3 and 4 are cumbersome.
	\item 3 hours.
\end{enumerate}

\newpage

%\includepdf[pages=-]{prob*.pdf}

\end{document}
