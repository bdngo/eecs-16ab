
    




    
\documentclass[11pt]{article}

    
    \usepackage[breakable]{tcolorbox}
    \tcbset{nobeforeafter} % prevents tcolorboxes being placing in paragraphs
    \usepackage{float}
    \floatplacement{figure}{H} % forces figures to be placed at the correct location
    
    \usepackage[T1]{fontenc}
    % Nicer default font (+ math font) than Computer Modern for most use cases
    \usepackage{mathpazo}

    % Basic figure setup, for now with no caption control since it's done
    % automatically by Pandoc (which extracts ![](path) syntax from Markdown).
    \usepackage{graphicx}
    % We will generate all images so they have a width \maxwidth. This means
    % that they will get their normal width if they fit onto the page, but
    % are scaled down if they would overflow the margins.
    \makeatletter
    \def\maxwidth{\ifdim\Gin@nat@width>\linewidth\linewidth
    \else\Gin@nat@width\fi}
    \makeatother
    \let\Oldincludegraphics\includegraphics
    % Set max figure width to be 80% of text width, for now hardcoded.
    \renewcommand{\includegraphics}[1]{\Oldincludegraphics[width=.8\maxwidth]{#1}}
    % Ensure that by default, figures have no caption (until we provide a
    % proper Figure object with a Caption API and a way to capture that
    % in the conversion process - todo).
    \usepackage{caption}
    \DeclareCaptionLabelFormat{nolabel}{}
    \captionsetup{labelformat=nolabel}

    \usepackage{adjustbox} % Used to constrain images to a maximum size 
    \usepackage{xcolor} % Allow colors to be defined
    \usepackage{enumerate} % Needed for markdown enumerations to work
    \usepackage{geometry} % Used to adjust the document margins
    \usepackage{amsmath} % Equations
    \usepackage{amssymb} % Equations
    \usepackage{textcomp} % defines textquotesingle
    % Hack from http://tex.stackexchange.com/a/47451/13684:
    \AtBeginDocument{%
        \def\PYZsq{\textquotesingle}% Upright quotes in Pygmentized code
    }
    \usepackage{upquote} % Upright quotes for verbatim code
    \usepackage{eurosym} % defines \euro
    \usepackage[mathletters]{ucs} % Extended unicode (utf-8) support
    \usepackage[utf8x]{inputenc} % Allow utf-8 characters in the tex document
    \usepackage{fancyvrb} % verbatim replacement that allows latex
    \usepackage{grffile} % extends the file name processing of package graphics 
                         % to support a larger range 
    % The hyperref package gives us a pdf with properly built
    % internal navigation ('pdf bookmarks' for the table of contents,
    % internal cross-reference links, web links for URLs, etc.)
    \usepackage{hyperref}
    \usepackage{longtable} % longtable support required by pandoc >1.10
    \usepackage{booktabs}  % table support for pandoc > 1.12.2
    \usepackage[inline]{enumitem} % IRkernel/repr support (it uses the enumerate* environment)
    \usepackage[normalem]{ulem} % ulem is needed to support strikethroughs (\sout)
                                % normalem makes italics be italics, not underlines
    \usepackage{mathrsfs}
    

    
    % Colors for the hyperref package
    \definecolor{urlcolor}{rgb}{0,.145,.698}
    \definecolor{linkcolor}{rgb}{.71,0.21,0.01}
    \definecolor{citecolor}{rgb}{.12,.54,.11}

    % ANSI colors
    \definecolor{ansi-black}{HTML}{3E424D}
    \definecolor{ansi-black-intense}{HTML}{282C36}
    \definecolor{ansi-red}{HTML}{E75C58}
    \definecolor{ansi-red-intense}{HTML}{B22B31}
    \definecolor{ansi-green}{HTML}{00A250}
    \definecolor{ansi-green-intense}{HTML}{007427}
    \definecolor{ansi-yellow}{HTML}{DDB62B}
    \definecolor{ansi-yellow-intense}{HTML}{B27D12}
    \definecolor{ansi-blue}{HTML}{208FFB}
    \definecolor{ansi-blue-intense}{HTML}{0065CA}
    \definecolor{ansi-magenta}{HTML}{D160C4}
    \definecolor{ansi-magenta-intense}{HTML}{A03196}
    \definecolor{ansi-cyan}{HTML}{60C6C8}
    \definecolor{ansi-cyan-intense}{HTML}{258F8F}
    \definecolor{ansi-white}{HTML}{C5C1B4}
    \definecolor{ansi-white-intense}{HTML}{A1A6B2}
    \definecolor{ansi-default-inverse-fg}{HTML}{FFFFFF}
    \definecolor{ansi-default-inverse-bg}{HTML}{000000}

    % commands and environments needed by pandoc snippets
    % extracted from the output of `pandoc -s`
    \providecommand{\tightlist}{%
      \setlength{\itemsep}{0pt}\setlength{\parskip}{0pt}}
    \DefineVerbatimEnvironment{Highlighting}{Verbatim}{commandchars=\\\{\}}
    % Add ',fontsize=\small' for more characters per line
    \newenvironment{Shaded}{}{}
    \newcommand{\KeywordTok}[1]{\textcolor[rgb]{0.00,0.44,0.13}{\textbf{{#1}}}}
    \newcommand{\DataTypeTok}[1]{\textcolor[rgb]{0.56,0.13,0.00}{{#1}}}
    \newcommand{\DecValTok}[1]{\textcolor[rgb]{0.25,0.63,0.44}{{#1}}}
    \newcommand{\BaseNTok}[1]{\textcolor[rgb]{0.25,0.63,0.44}{{#1}}}
    \newcommand{\FloatTok}[1]{\textcolor[rgb]{0.25,0.63,0.44}{{#1}}}
    \newcommand{\CharTok}[1]{\textcolor[rgb]{0.25,0.44,0.63}{{#1}}}
    \newcommand{\StringTok}[1]{\textcolor[rgb]{0.25,0.44,0.63}{{#1}}}
    \newcommand{\CommentTok}[1]{\textcolor[rgb]{0.38,0.63,0.69}{\textit{{#1}}}}
    \newcommand{\OtherTok}[1]{\textcolor[rgb]{0.00,0.44,0.13}{{#1}}}
    \newcommand{\AlertTok}[1]{\textcolor[rgb]{1.00,0.00,0.00}{\textbf{{#1}}}}
    \newcommand{\FunctionTok}[1]{\textcolor[rgb]{0.02,0.16,0.49}{{#1}}}
    \newcommand{\RegionMarkerTok}[1]{{#1}}
    \newcommand{\ErrorTok}[1]{\textcolor[rgb]{1.00,0.00,0.00}{\textbf{{#1}}}}
    \newcommand{\NormalTok}[1]{{#1}}
    
    % Additional commands for more recent versions of Pandoc
    \newcommand{\ConstantTok}[1]{\textcolor[rgb]{0.53,0.00,0.00}{{#1}}}
    \newcommand{\SpecialCharTok}[1]{\textcolor[rgb]{0.25,0.44,0.63}{{#1}}}
    \newcommand{\VerbatimStringTok}[1]{\textcolor[rgb]{0.25,0.44,0.63}{{#1}}}
    \newcommand{\SpecialStringTok}[1]{\textcolor[rgb]{0.73,0.40,0.53}{{#1}}}
    \newcommand{\ImportTok}[1]{{#1}}
    \newcommand{\DocumentationTok}[1]{\textcolor[rgb]{0.73,0.13,0.13}{\textit{{#1}}}}
    \newcommand{\AnnotationTok}[1]{\textcolor[rgb]{0.38,0.63,0.69}{\textbf{\textit{{#1}}}}}
    \newcommand{\CommentVarTok}[1]{\textcolor[rgb]{0.38,0.63,0.69}{\textbf{\textit{{#1}}}}}
    \newcommand{\VariableTok}[1]{\textcolor[rgb]{0.10,0.09,0.49}{{#1}}}
    \newcommand{\ControlFlowTok}[1]{\textcolor[rgb]{0.00,0.44,0.13}{\textbf{{#1}}}}
    \newcommand{\OperatorTok}[1]{\textcolor[rgb]{0.40,0.40,0.40}{{#1}}}
    \newcommand{\BuiltInTok}[1]{{#1}}
    \newcommand{\ExtensionTok}[1]{{#1}}
    \newcommand{\PreprocessorTok}[1]{\textcolor[rgb]{0.74,0.48,0.00}{{#1}}}
    \newcommand{\AttributeTok}[1]{\textcolor[rgb]{0.49,0.56,0.16}{{#1}}}
    \newcommand{\InformationTok}[1]{\textcolor[rgb]{0.38,0.63,0.69}{\textbf{\textit{{#1}}}}}
    \newcommand{\WarningTok}[1]{\textcolor[rgb]{0.38,0.63,0.69}{\textbf{\textit{{#1}}}}}
    
    
    % Define a nice break command that doesn't care if a line doesn't already
    % exist.
    \def\br{\hspace*{\fill} \\* }
    % Math Jax compatibility definitions
    \def\gt{>}
    \def\lt{<}
    \let\Oldtex\TeX
    \let\Oldlatex\LaTeX
    \renewcommand{\TeX}{\textrm{\Oldtex}}
    \renewcommand{\LaTeX}{\textrm{\Oldlatex}}
    % Document parameters
    % Document title
    \title{prob12}
    
    
    
    
    
% Pygments definitions
\makeatletter
\def\PY@reset{\let\PY@it=\relax \let\PY@bf=\relax%
    \let\PY@ul=\relax \let\PY@tc=\relax%
    \let\PY@bc=\relax \let\PY@ff=\relax}
\def\PY@tok#1{\csname PY@tok@#1\endcsname}
\def\PY@toks#1+{\ifx\relax#1\empty\else%
    \PY@tok{#1}\expandafter\PY@toks\fi}
\def\PY@do#1{\PY@bc{\PY@tc{\PY@ul{%
    \PY@it{\PY@bf{\PY@ff{#1}}}}}}}
\def\PY#1#2{\PY@reset\PY@toks#1+\relax+\PY@do{#2}}

\expandafter\def\csname PY@tok@w\endcsname{\def\PY@tc##1{\textcolor[rgb]{0.73,0.73,0.73}{##1}}}
\expandafter\def\csname PY@tok@c\endcsname{\let\PY@it=\textit\def\PY@tc##1{\textcolor[rgb]{0.25,0.50,0.50}{##1}}}
\expandafter\def\csname PY@tok@cp\endcsname{\def\PY@tc##1{\textcolor[rgb]{0.74,0.48,0.00}{##1}}}
\expandafter\def\csname PY@tok@k\endcsname{\let\PY@bf=\textbf\def\PY@tc##1{\textcolor[rgb]{0.00,0.50,0.00}{##1}}}
\expandafter\def\csname PY@tok@kp\endcsname{\def\PY@tc##1{\textcolor[rgb]{0.00,0.50,0.00}{##1}}}
\expandafter\def\csname PY@tok@kt\endcsname{\def\PY@tc##1{\textcolor[rgb]{0.69,0.00,0.25}{##1}}}
\expandafter\def\csname PY@tok@o\endcsname{\def\PY@tc##1{\textcolor[rgb]{0.40,0.40,0.40}{##1}}}
\expandafter\def\csname PY@tok@ow\endcsname{\let\PY@bf=\textbf\def\PY@tc##1{\textcolor[rgb]{0.67,0.13,1.00}{##1}}}
\expandafter\def\csname PY@tok@nb\endcsname{\def\PY@tc##1{\textcolor[rgb]{0.00,0.50,0.00}{##1}}}
\expandafter\def\csname PY@tok@nf\endcsname{\def\PY@tc##1{\textcolor[rgb]{0.00,0.00,1.00}{##1}}}
\expandafter\def\csname PY@tok@nc\endcsname{\let\PY@bf=\textbf\def\PY@tc##1{\textcolor[rgb]{0.00,0.00,1.00}{##1}}}
\expandafter\def\csname PY@tok@nn\endcsname{\let\PY@bf=\textbf\def\PY@tc##1{\textcolor[rgb]{0.00,0.00,1.00}{##1}}}
\expandafter\def\csname PY@tok@ne\endcsname{\let\PY@bf=\textbf\def\PY@tc##1{\textcolor[rgb]{0.82,0.25,0.23}{##1}}}
\expandafter\def\csname PY@tok@nv\endcsname{\def\PY@tc##1{\textcolor[rgb]{0.10,0.09,0.49}{##1}}}
\expandafter\def\csname PY@tok@no\endcsname{\def\PY@tc##1{\textcolor[rgb]{0.53,0.00,0.00}{##1}}}
\expandafter\def\csname PY@tok@nl\endcsname{\def\PY@tc##1{\textcolor[rgb]{0.63,0.63,0.00}{##1}}}
\expandafter\def\csname PY@tok@ni\endcsname{\let\PY@bf=\textbf\def\PY@tc##1{\textcolor[rgb]{0.60,0.60,0.60}{##1}}}
\expandafter\def\csname PY@tok@na\endcsname{\def\PY@tc##1{\textcolor[rgb]{0.49,0.56,0.16}{##1}}}
\expandafter\def\csname PY@tok@nt\endcsname{\let\PY@bf=\textbf\def\PY@tc##1{\textcolor[rgb]{0.00,0.50,0.00}{##1}}}
\expandafter\def\csname PY@tok@nd\endcsname{\def\PY@tc##1{\textcolor[rgb]{0.67,0.13,1.00}{##1}}}
\expandafter\def\csname PY@tok@s\endcsname{\def\PY@tc##1{\textcolor[rgb]{0.73,0.13,0.13}{##1}}}
\expandafter\def\csname PY@tok@sd\endcsname{\let\PY@it=\textit\def\PY@tc##1{\textcolor[rgb]{0.73,0.13,0.13}{##1}}}
\expandafter\def\csname PY@tok@si\endcsname{\let\PY@bf=\textbf\def\PY@tc##1{\textcolor[rgb]{0.73,0.40,0.53}{##1}}}
\expandafter\def\csname PY@tok@se\endcsname{\let\PY@bf=\textbf\def\PY@tc##1{\textcolor[rgb]{0.73,0.40,0.13}{##1}}}
\expandafter\def\csname PY@tok@sr\endcsname{\def\PY@tc##1{\textcolor[rgb]{0.73,0.40,0.53}{##1}}}
\expandafter\def\csname PY@tok@ss\endcsname{\def\PY@tc##1{\textcolor[rgb]{0.10,0.09,0.49}{##1}}}
\expandafter\def\csname PY@tok@sx\endcsname{\def\PY@tc##1{\textcolor[rgb]{0.00,0.50,0.00}{##1}}}
\expandafter\def\csname PY@tok@m\endcsname{\def\PY@tc##1{\textcolor[rgb]{0.40,0.40,0.40}{##1}}}
\expandafter\def\csname PY@tok@gh\endcsname{\let\PY@bf=\textbf\def\PY@tc##1{\textcolor[rgb]{0.00,0.00,0.50}{##1}}}
\expandafter\def\csname PY@tok@gu\endcsname{\let\PY@bf=\textbf\def\PY@tc##1{\textcolor[rgb]{0.50,0.00,0.50}{##1}}}
\expandafter\def\csname PY@tok@gd\endcsname{\def\PY@tc##1{\textcolor[rgb]{0.63,0.00,0.00}{##1}}}
\expandafter\def\csname PY@tok@gi\endcsname{\def\PY@tc##1{\textcolor[rgb]{0.00,0.63,0.00}{##1}}}
\expandafter\def\csname PY@tok@gr\endcsname{\def\PY@tc##1{\textcolor[rgb]{1.00,0.00,0.00}{##1}}}
\expandafter\def\csname PY@tok@ge\endcsname{\let\PY@it=\textit}
\expandafter\def\csname PY@tok@gs\endcsname{\let\PY@bf=\textbf}
\expandafter\def\csname PY@tok@gp\endcsname{\let\PY@bf=\textbf\def\PY@tc##1{\textcolor[rgb]{0.00,0.00,0.50}{##1}}}
\expandafter\def\csname PY@tok@go\endcsname{\def\PY@tc##1{\textcolor[rgb]{0.53,0.53,0.53}{##1}}}
\expandafter\def\csname PY@tok@gt\endcsname{\def\PY@tc##1{\textcolor[rgb]{0.00,0.27,0.87}{##1}}}
\expandafter\def\csname PY@tok@err\endcsname{\def\PY@bc##1{\setlength{\fboxsep}{0pt}\fcolorbox[rgb]{1.00,0.00,0.00}{1,1,1}{\strut ##1}}}
\expandafter\def\csname PY@tok@kc\endcsname{\let\PY@bf=\textbf\def\PY@tc##1{\textcolor[rgb]{0.00,0.50,0.00}{##1}}}
\expandafter\def\csname PY@tok@kd\endcsname{\let\PY@bf=\textbf\def\PY@tc##1{\textcolor[rgb]{0.00,0.50,0.00}{##1}}}
\expandafter\def\csname PY@tok@kn\endcsname{\let\PY@bf=\textbf\def\PY@tc##1{\textcolor[rgb]{0.00,0.50,0.00}{##1}}}
\expandafter\def\csname PY@tok@kr\endcsname{\let\PY@bf=\textbf\def\PY@tc##1{\textcolor[rgb]{0.00,0.50,0.00}{##1}}}
\expandafter\def\csname PY@tok@bp\endcsname{\def\PY@tc##1{\textcolor[rgb]{0.00,0.50,0.00}{##1}}}
\expandafter\def\csname PY@tok@fm\endcsname{\def\PY@tc##1{\textcolor[rgb]{0.00,0.00,1.00}{##1}}}
\expandafter\def\csname PY@tok@vc\endcsname{\def\PY@tc##1{\textcolor[rgb]{0.10,0.09,0.49}{##1}}}
\expandafter\def\csname PY@tok@vg\endcsname{\def\PY@tc##1{\textcolor[rgb]{0.10,0.09,0.49}{##1}}}
\expandafter\def\csname PY@tok@vi\endcsname{\def\PY@tc##1{\textcolor[rgb]{0.10,0.09,0.49}{##1}}}
\expandafter\def\csname PY@tok@vm\endcsname{\def\PY@tc##1{\textcolor[rgb]{0.10,0.09,0.49}{##1}}}
\expandafter\def\csname PY@tok@sa\endcsname{\def\PY@tc##1{\textcolor[rgb]{0.73,0.13,0.13}{##1}}}
\expandafter\def\csname PY@tok@sb\endcsname{\def\PY@tc##1{\textcolor[rgb]{0.73,0.13,0.13}{##1}}}
\expandafter\def\csname PY@tok@sc\endcsname{\def\PY@tc##1{\textcolor[rgb]{0.73,0.13,0.13}{##1}}}
\expandafter\def\csname PY@tok@dl\endcsname{\def\PY@tc##1{\textcolor[rgb]{0.73,0.13,0.13}{##1}}}
\expandafter\def\csname PY@tok@s2\endcsname{\def\PY@tc##1{\textcolor[rgb]{0.73,0.13,0.13}{##1}}}
\expandafter\def\csname PY@tok@sh\endcsname{\def\PY@tc##1{\textcolor[rgb]{0.73,0.13,0.13}{##1}}}
\expandafter\def\csname PY@tok@s1\endcsname{\def\PY@tc##1{\textcolor[rgb]{0.73,0.13,0.13}{##1}}}
\expandafter\def\csname PY@tok@mb\endcsname{\def\PY@tc##1{\textcolor[rgb]{0.40,0.40,0.40}{##1}}}
\expandafter\def\csname PY@tok@mf\endcsname{\def\PY@tc##1{\textcolor[rgb]{0.40,0.40,0.40}{##1}}}
\expandafter\def\csname PY@tok@mh\endcsname{\def\PY@tc##1{\textcolor[rgb]{0.40,0.40,0.40}{##1}}}
\expandafter\def\csname PY@tok@mi\endcsname{\def\PY@tc##1{\textcolor[rgb]{0.40,0.40,0.40}{##1}}}
\expandafter\def\csname PY@tok@il\endcsname{\def\PY@tc##1{\textcolor[rgb]{0.40,0.40,0.40}{##1}}}
\expandafter\def\csname PY@tok@mo\endcsname{\def\PY@tc##1{\textcolor[rgb]{0.40,0.40,0.40}{##1}}}
\expandafter\def\csname PY@tok@ch\endcsname{\let\PY@it=\textit\def\PY@tc##1{\textcolor[rgb]{0.25,0.50,0.50}{##1}}}
\expandafter\def\csname PY@tok@cm\endcsname{\let\PY@it=\textit\def\PY@tc##1{\textcolor[rgb]{0.25,0.50,0.50}{##1}}}
\expandafter\def\csname PY@tok@cpf\endcsname{\let\PY@it=\textit\def\PY@tc##1{\textcolor[rgb]{0.25,0.50,0.50}{##1}}}
\expandafter\def\csname PY@tok@c1\endcsname{\let\PY@it=\textit\def\PY@tc##1{\textcolor[rgb]{0.25,0.50,0.50}{##1}}}
\expandafter\def\csname PY@tok@cs\endcsname{\let\PY@it=\textit\def\PY@tc##1{\textcolor[rgb]{0.25,0.50,0.50}{##1}}}

\def\PYZbs{\char`\\}
\def\PYZus{\char`\_}
\def\PYZob{\char`\{}
\def\PYZcb{\char`\}}
\def\PYZca{\char`\^}
\def\PYZam{\char`\&}
\def\PYZlt{\char`\<}
\def\PYZgt{\char`\>}
\def\PYZsh{\char`\#}
\def\PYZpc{\char`\%}
\def\PYZdl{\char`\$}
\def\PYZhy{\char`\-}
\def\PYZsq{\char`\'}
\def\PYZdq{\char`\"}
\def\PYZti{\char`\~}
% for compatibility with earlier versions
\def\PYZat{@}
\def\PYZlb{[}
\def\PYZrb{]}
\makeatother


    % For linebreaks inside Verbatim environment from package fancyvrb. 
    \makeatletter
        \newbox\Wrappedcontinuationbox 
        \newbox\Wrappedvisiblespacebox 
        \newcommand*\Wrappedvisiblespace {\textcolor{red}{\textvisiblespace}} 
        \newcommand*\Wrappedcontinuationsymbol {\textcolor{red}{\llap{\tiny$\m@th\hookrightarrow$}}} 
        \newcommand*\Wrappedcontinuationindent {3ex } 
        \newcommand*\Wrappedafterbreak {\kern\Wrappedcontinuationindent\copy\Wrappedcontinuationbox} 
        % Take advantage of the already applied Pygments mark-up to insert 
        % potential linebreaks for TeX processing. 
        %        {, <, #, %, $, ' and ": go to next line. 
        %        _, }, ^, &, >, - and ~: stay at end of broken line. 
        % Use of \textquotesingle for straight quote. 
        \newcommand*\Wrappedbreaksatspecials {% 
            \def\PYGZus{\discretionary{\char`\_}{\Wrappedafterbreak}{\char`\_}}% 
            \def\PYGZob{\discretionary{}{\Wrappedafterbreak\char`\{}{\char`\{}}% 
            \def\PYGZcb{\discretionary{\char`\}}{\Wrappedafterbreak}{\char`\}}}% 
            \def\PYGZca{\discretionary{\char`\^}{\Wrappedafterbreak}{\char`\^}}% 
            \def\PYGZam{\discretionary{\char`\&}{\Wrappedafterbreak}{\char`\&}}% 
            \def\PYGZlt{\discretionary{}{\Wrappedafterbreak\char`\<}{\char`\<}}% 
            \def\PYGZgt{\discretionary{\char`\>}{\Wrappedafterbreak}{\char`\>}}% 
            \def\PYGZsh{\discretionary{}{\Wrappedafterbreak\char`\#}{\char`\#}}% 
            \def\PYGZpc{\discretionary{}{\Wrappedafterbreak\char`\%}{\char`\%}}% 
            \def\PYGZdl{\discretionary{}{\Wrappedafterbreak\char`\$}{\char`\$}}% 
            \def\PYGZhy{\discretionary{\char`\-}{\Wrappedafterbreak}{\char`\-}}% 
            \def\PYGZsq{\discretionary{}{\Wrappedafterbreak\textquotesingle}{\textquotesingle}}% 
            \def\PYGZdq{\discretionary{}{\Wrappedafterbreak\char`\"}{\char`\"}}% 
            \def\PYGZti{\discretionary{\char`\~}{\Wrappedafterbreak}{\char`\~}}% 
        } 
        % Some characters . , ; ? ! / are not pygmentized. 
        % This macro makes them "active" and they will insert potential linebreaks 
        \newcommand*\Wrappedbreaksatpunct {% 
            \lccode`\~`\.\lowercase{\def~}{\discretionary{\hbox{\char`\.}}{\Wrappedafterbreak}{\hbox{\char`\.}}}% 
            \lccode`\~`\,\lowercase{\def~}{\discretionary{\hbox{\char`\,}}{\Wrappedafterbreak}{\hbox{\char`\,}}}% 
            \lccode`\~`\;\lowercase{\def~}{\discretionary{\hbox{\char`\;}}{\Wrappedafterbreak}{\hbox{\char`\;}}}% 
            \lccode`\~`\:\lowercase{\def~}{\discretionary{\hbox{\char`\:}}{\Wrappedafterbreak}{\hbox{\char`\:}}}% 
            \lccode`\~`\?\lowercase{\def~}{\discretionary{\hbox{\char`\?}}{\Wrappedafterbreak}{\hbox{\char`\?}}}% 
            \lccode`\~`\!\lowercase{\def~}{\discretionary{\hbox{\char`\!}}{\Wrappedafterbreak}{\hbox{\char`\!}}}% 
            \lccode`\~`\/\lowercase{\def~}{\discretionary{\hbox{\char`\/}}{\Wrappedafterbreak}{\hbox{\char`\/}}}% 
            \catcode`\.\active
            \catcode`\,\active 
            \catcode`\;\active
            \catcode`\:\active
            \catcode`\?\active
            \catcode`\!\active
            \catcode`\/\active 
            \lccode`\~`\~ 	
        }
    \makeatother

    \let\OriginalVerbatim=\Verbatim
    \makeatletter
    \renewcommand{\Verbatim}[1][1]{%
        %\parskip\z@skip
        \sbox\Wrappedcontinuationbox {\Wrappedcontinuationsymbol}%
        \sbox\Wrappedvisiblespacebox {\FV@SetupFont\Wrappedvisiblespace}%
        \def\FancyVerbFormatLine ##1{\hsize\linewidth
            \vtop{\raggedright\hyphenpenalty\z@\exhyphenpenalty\z@
                \doublehyphendemerits\z@\finalhyphendemerits\z@
                \strut ##1\strut}%
        }%
        % If the linebreak is at a space, the latter will be displayed as visible
        % space at end of first line, and a continuation symbol starts next line.
        % Stretch/shrink are however usually zero for typewriter font.
        \def\FV@Space {%
            \nobreak\hskip\z@ plus\fontdimen3\font minus\fontdimen4\font
            \discretionary{\copy\Wrappedvisiblespacebox}{\Wrappedafterbreak}
            {\kern\fontdimen2\font}%
        }%
        
        % Allow breaks at special characters using \PYG... macros.
        \Wrappedbreaksatspecials
        % Breaks at punctuation characters . , ; ? ! and / need catcode=\active 	
        \OriginalVerbatim[#1,codes*=\Wrappedbreaksatpunct]%
    }
    \makeatother

    % Exact colors from NB
    \definecolor{incolor}{HTML}{303F9F}
    \definecolor{outcolor}{HTML}{D84315}
    \definecolor{cellborder}{HTML}{CFCFCF}
    \definecolor{cellbackground}{HTML}{F7F7F7}
    
    % prompt
    \newcommand{\prompt}[4]{
        \llap{{\color{#2}[#3]: #4}}\vspace{-1.25em}
    }
    

    
    % Prevent overflowing lines due to hard-to-break entities
    \sloppy 
    % Setup hyperref package
    \hypersetup{
      breaklinks=true,  % so long urls are correctly broken across lines
      colorlinks=true,
      urlcolor=urlcolor,
      linkcolor=linkcolor,
      citecolor=citecolor,
      }
    % Slightly bigger margins than the latex defaults
    
    \geometry{verbose,tmargin=1in,bmargin=1in,lmargin=1in,rmargin=1in}
    
    

    \begin{document}
    
    
    \maketitle
    
    

    
    \hypertarget{eecs16a-homework-12}{%
\section{EECS16A Homework 12}\label{eecs16a-homework-12}}

    \hypertarget{question-3-mechanical-correlation}{%
\subsection{Question 3: Mechanical
Correlation}\label{question-3-mechanical-correlation}}

    \hypertarget{part-c}{%
\subsubsection{Part (c)}\label{part-c}}

    \begin{tcolorbox}[breakable, size=fbox, boxrule=1pt, pad at break*=1mm,colback=cellbackground, colframe=cellborder]
\prompt{In}{incolor}{1}{\hspace{4pt}}
\begin{Verbatim}[commandchars=\\\{\}]
\PY{k+kn}{import} \PY{n+nn}{numpy} \PY{k}{as} \PY{n+nn}{np}
\PY{n}{s1} \PY{o}{=} \PY{p}{[}\PY{l+m+mi}{2}\PY{p}{,} \PY{o}{\PYZhy{}}\PY{l+m+mi}{2}\PY{p}{,} \PY{l+m+mi}{2}\PY{p}{,} \PY{o}{\PYZhy{}}\PY{l+m+mi}{2}\PY{p}{]}
\PY{n}{s2} \PY{o}{=} \PY{p}{[}\PY{l+m+mi}{1}\PY{p}{,} \PY{l+m+mi}{2}\PY{p}{,} \PY{l+m+mi}{3}\PY{p}{,} \PY{l+m+mi}{4}\PY{p}{]}

\PY{c+c1}{\PYZsh{} Use the function np.correlate with mode=\PYZsq{}full\PYZsq{} for linear cross correlation.}
\PY{c+c1}{\PYZsh{}\PYZsh{} your code here}
\PY{n+nb}{print}\PY{p}{(}\PY{n}{np}\PY{o}{.}\PY{n}{correlate}\PY{p}{(}\PY{n}{s1}\PY{p}{,} \PY{n}{s2}\PY{p}{,} \PY{l+s+s1}{\PYZsq{}}\PY{l+s+s1}{full}\PY{l+s+s1}{\PYZsq{}}\PY{p}{)}\PY{p}{)}
\PY{n+nb}{print}\PY{p}{(}\PY{n}{np}\PY{o}{.}\PY{n}{correlate}\PY{p}{(}\PY{n}{s2}\PY{p}{,} \PY{n}{s1}\PY{p}{,} \PY{l+s+s1}{\PYZsq{}}\PY{l+s+s1}{full}\PY{l+s+s1}{\PYZsq{}}\PY{p}{)}\PY{p}{)}
\end{Verbatim}
\end{tcolorbox}

    \begin{Verbatim}[commandchars=\\\{\}]
[ 8 -2  6 -4 -4 -2 -2]
[-2 -2 -4 -4  6 -2  8]
\end{Verbatim}

    \hypertarget{question-4-audio-file-matching}{%
\subsection{Question 4: Audio File
Matching}\label{question-4-audio-file-matching}}

This notebook continues the audio file matching problem. Be sure to have
song.wav and clip.wav in the same directory as the notebook.

In this notebook, we will look at the problem of searching for a small
audio clip inside a song.

The song ``Mandelbrot Set'' by Jonathan Coulton is licensed under CC
BY-NC 3.0

If you have trouble playing the audio file in IPython, try opening it in
a different browser. I encountered problem with Safari but Chrome works
for me.

    \begin{tcolorbox}[breakable, size=fbox, boxrule=1pt, pad at break*=1mm,colback=cellbackground, colframe=cellborder]
\prompt{In}{incolor}{ }{\hspace{4pt}}
\begin{Verbatim}[commandchars=\\\{\}]
\PY{k+kn}{import} \PY{n+nn}{numpy} \PY{k}{as} \PY{n+nn}{np}
\PY{k+kn}{import} \PY{n+nn}{wave}
\PY{k+kn}{import} \PY{n+nn}{matplotlib}\PY{n+nn}{.}\PY{n+nn}{pyplot} \PY{k}{as} \PY{n+nn}{plt}
\PY{k+kn}{import} \PY{n+nn}{scipy}\PY{n+nn}{.}\PY{n+nn}{io}\PY{n+nn}{.}\PY{n+nn}{wavfile}
\PY{k+kn}{import} \PY{n+nn}{operator}
\PY{k+kn}{from} \PY{n+nn}{IPython}\PY{n+nn}{.}\PY{n+nn}{display} \PY{k}{import} \PY{n}{Audio}
\PY{o}{\PYZpc{}}\PY{k}{matplotlib} inline

\PY{n}{given\PYZus{}file} \PY{o}{=} \PY{l+s+s1}{\PYZsq{}}\PY{l+s+s1}{song.wav}\PY{l+s+s1}{\PYZsq{}}
\PY{n}{target\PYZus{}file} \PY{o}{=} \PY{l+s+s1}{\PYZsq{}}\PY{l+s+s1}{clip.wav}\PY{l+s+s1}{\PYZsq{}}
\PY{n}{rate\PYZus{}given}\PY{p}{,}  \PY{n}{given\PYZus{}signal}  \PY{o}{=} \PY{n}{scipy}\PY{o}{.}\PY{n}{io}\PY{o}{.}\PY{n}{wavfile}\PY{o}{.}\PY{n}{read}\PY{p}{(}\PY{n}{given\PYZus{}file}\PY{p}{)}
\PY{n}{rate\PYZus{}target}\PY{p}{,} \PY{n}{target\PYZus{}signal} \PY{o}{=} \PY{n}{scipy}\PY{o}{.}\PY{n}{io}\PY{o}{.}\PY{n}{wavfile}\PY{o}{.}\PY{n}{read}\PY{p}{(}\PY{n}{target\PYZus{}file}\PY{p}{)}
\PY{n}{given\PYZus{}signal}  \PY{o}{=} \PY{n}{given\PYZus{}signal}\PY{p}{[}\PY{p}{:}\PY{l+m+mi}{2000000}\PY{p}{]}\PY{o}{.}\PY{n}{astype}\PY{p}{(}\PY{n+nb}{float}\PY{p}{)}
\PY{n}{target\PYZus{}signal} \PY{o}{=} \PY{n}{target\PYZus{}signal}\PY{o}{.}\PY{n}{astype}\PY{p}{(}\PY{n+nb}{float}\PY{p}{)}
\PY{k}{def} \PY{n+nf}{play\PYZus{}clip}\PY{p}{(}\PY{n}{start}\PY{p}{,} \PY{n}{end}\PY{p}{,} \PY{n}{signal}\PY{o}{=}\PY{n}{given\PYZus{}signal}\PY{p}{)}\PY{p}{:}
    \PY{n}{scipy}\PY{o}{.}\PY{n}{io}\PY{o}{.}\PY{n}{wavfile}\PY{o}{.}\PY{n}{write}\PY{p}{(}\PY{l+s+s1}{\PYZsq{}}\PY{l+s+s1}{temp.wav}\PY{l+s+s1}{\PYZsq{}}\PY{p}{,} \PY{n}{rate\PYZus{}given}\PY{p}{,} \PY{n}{signal}\PY{p}{[}\PY{n}{start}\PY{p}{:}\PY{n}{end}\PY{p}{]}\PY{o}{.}\PY{n}{astype}\PY{p}{(}\PY{n}{np}\PY{o}{.}\PY{n}{int16}\PY{p}{)}\PY{p}{)}
    \PY{k}{return} \PY{n}{Audio}\PY{p}{(}\PY{n}{url}\PY{o}{=}\PY{l+s+s1}{\PYZsq{}}\PY{l+s+s1}{temp.wav}\PY{l+s+s1}{\PYZsq{}}\PY{p}{,} \PY{n}{autoplay}\PY{o}{=}\PY{k+kc}{True}\PY{p}{)}

\PY{k}{def} \PY{n+nf}{run\PYZus{}comparison}\PY{p}{(}\PY{n}{target\PYZus{}signal}\PY{p}{,} \PY{n}{given\PYZus{}signal}\PY{p}{,} \PY{n}{idxs}\PY{o}{=}\PY{k+kc}{None}\PY{p}{)}\PY{p}{:}
    \PY{c+c1}{\PYZsh{} Run everything if not called with idxs set to something}
    \PY{k}{if} \PY{n}{idxs} \PY{o+ow}{is} \PY{k+kc}{None}\PY{p}{:}
        \PY{n}{idxs} \PY{o}{=} \PY{p}{[}\PY{n}{i} \PY{k}{for} \PY{n}{i} \PY{o+ow}{in} \PY{n+nb}{range}\PY{p}{(}\PY{n+nb}{len}\PY{p}{(}\PY{n}{given\PYZus{}signal}\PY{p}{)}\PY{o}{\PYZhy{}}\PY{n+nb}{len}\PY{p}{(}\PY{n}{target\PYZus{}signal}\PY{p}{)}\PY{p}{)}\PY{p}{]}
    \PY{k}{return} \PY{n}{idxs}\PY{p}{,} \PY{p}{[}\PY{n}{vector\PYZus{}compare}\PY{p}{(}\PY{n}{target\PYZus{}signal}\PY{p}{,} \PY{n}{given\PYZus{}signal}\PY{p}{[}\PY{n}{i}\PY{p}{:}\PY{n}{i}\PY{o}{+}\PY{n+nb}{len}\PY{p}{(}\PY{n}{target\PYZus{}signal}\PY{p}{)}\PY{p}{]}\PY{p}{)}
                \PY{k}{for} \PY{n}{i} \PY{o+ow}{in} \PY{n}{idxs}\PY{p}{]}

\PY{n}{play\PYZus{}clip}\PY{p}{(}\PY{l+m+mi}{0}\PY{p}{,} \PY{n+nb}{len}\PY{p}{(}\PY{n}{given\PYZus{}signal}\PY{p}{)}\PY{p}{)}
\end{Verbatim}
\end{tcolorbox}

    We will load the song into the variable \texttt{given\_signal} and load
the short clip into the variable \texttt{target\_signal}. Your job is to
finish code that will identify the short clip's location in the song.
The clip we are trying to find will play after executing the following
block.

    \begin{tcolorbox}[breakable, size=fbox, boxrule=1pt, pad at break*=1mm,colback=cellbackground, colframe=cellborder]
\prompt{In}{incolor}{ }{\hspace{4pt}}
\begin{Verbatim}[commandchars=\\\{\}]
\PY{n}{Audio}\PY{p}{(}\PY{n}{url}\PY{o}{=}\PY{n}{target\PYZus{}file}\PY{p}{,} \PY{n}{autoplay}\PY{o}{=}\PY{k+kc}{True}\PY{p}{)}
\end{Verbatim}
\end{tcolorbox}

    Your task is to define the function `vector\_compare' and run the
following code. Because the song has a lot of data, you should use the
provided examples from the previous parts of the problem before running
the later code. Do you results here make sense given your answers to
previous parts of the problem?

    \begin{tcolorbox}[breakable, size=fbox, boxrule=1pt, pad at break*=1mm,colback=cellbackground, colframe=cellborder]
\prompt{In}{incolor}{ }{\hspace{4pt}}
\begin{Verbatim}[commandchars=\\\{\}]
\PY{k}{def} \PY{n+nf}{vector\PYZus{}compare}\PY{p}{(}\PY{n}{short\PYZus{}clip}\PY{p}{,} \PY{n}{segment\PYZus{}of\PYZus{}song}\PY{p}{)}\PY{p}{:}
    \PY{l+s+sd}{\PYZdq{}\PYZdq{}\PYZdq{}This function compares two vectors, returning a number.}
\PY{l+s+sd}{    The test vector with the highest return value is regarded as being closest to the desired vector.\PYZdq{}\PYZdq{}\PYZdq{}}
    \PY{k}{return} \PY{c+c1}{\PYZsh{}\PYZsh{} your code here}

\PY{n+nb}{print}\PY{p}{(}\PY{l+s+s2}{\PYZdq{}}\PY{l+s+s2}{PART A:}\PY{l+s+s2}{\PYZdq{}}\PY{p}{)}
\PY{n+nb}{print}\PY{p}{(}\PY{n}{vector\PYZus{}compare}\PY{p}{(}\PY{n}{np}\PY{o}{.}\PY{n}{array}\PY{p}{(}\PY{p}{[}\PY{l+m+mi}{1}\PY{p}{,}\PY{l+m+mi}{1}\PY{p}{,}\PY{l+m+mi}{1}\PY{p}{]}\PY{p}{)}\PY{p}{,} \PY{n}{np}\PY{o}{.}\PY{n}{array}\PY{p}{(}\PY{p}{[}\PY{l+m+mi}{1}\PY{p}{,}\PY{l+m+mi}{1}\PY{p}{,}\PY{l+m+mi}{1}\PY{p}{]}\PY{p}{)}\PY{p}{)}\PY{p}{)}
\PY{n+nb}{print}\PY{p}{(}\PY{n}{vector\PYZus{}compare}\PY{p}{(}\PY{n}{np}\PY{o}{.}\PY{n}{array}\PY{p}{(}\PY{p}{[}\PY{l+m+mi}{1}\PY{p}{,}\PY{l+m+mi}{1}\PY{p}{,}\PY{l+m+mi}{1}\PY{p}{]}\PY{p}{)}\PY{p}{,} \PY{n}{np}\PY{o}{.}\PY{n}{array}\PY{p}{(}\PY{p}{[}\PY{o}{\PYZhy{}}\PY{l+m+mi}{1}\PY{p}{,}\PY{o}{\PYZhy{}}\PY{l+m+mi}{1}\PY{p}{,}\PY{o}{\PYZhy{}}\PY{l+m+mi}{1}\PY{p}{]}\PY{p}{)}\PY{p}{)}\PY{p}{)}
\PY{n+nb}{print}\PY{p}{(}\PY{l+s+s2}{\PYZdq{}}\PY{l+s+s2}{PART C:}\PY{l+s+s2}{\PYZdq{}}\PY{p}{)}
\PY{n+nb}{print}\PY{p}{(}\PY{n}{vector\PYZus{}compare}\PY{p}{(}\PY{n}{np}\PY{o}{.}\PY{n}{array}\PY{p}{(}\PY{p}{[}\PY{l+m+mi}{1}\PY{p}{,}\PY{l+m+mi}{2}\PY{p}{,}\PY{l+m+mi}{3}\PY{p}{]}\PY{p}{)}\PY{p}{,} \PY{n}{np}\PY{o}{.}\PY{n}{array}\PY{p}{(}\PY{p}{[}\PY{l+m+mi}{1}\PY{p}{,}\PY{l+m+mi}{2}\PY{p}{,}\PY{l+m+mi}{3}\PY{p}{]}\PY{p}{)}\PY{p}{)}\PY{p}{)}
\PY{n+nb}{print}\PY{p}{(}\PY{n}{vector\PYZus{}compare}\PY{p}{(}\PY{n}{np}\PY{o}{.}\PY{n}{array}\PY{p}{(}\PY{p}{[}\PY{l+m+mi}{1}\PY{p}{,}\PY{l+m+mi}{2}\PY{p}{,}\PY{l+m+mi}{3}\PY{p}{]}\PY{p}{)}\PY{p}{,} \PY{n}{np}\PY{o}{.}\PY{n}{array}\PY{p}{(}\PY{p}{[}\PY{l+m+mi}{2}\PY{p}{,}\PY{l+m+mi}{3}\PY{p}{,}\PY{l+m+mi}{4}\PY{p}{]}\PY{p}{)}\PY{p}{)}\PY{p}{)}
\PY{n+nb}{print}\PY{p}{(}\PY{n}{vector\PYZus{}compare}\PY{p}{(}\PY{n}{np}\PY{o}{.}\PY{n}{array}\PY{p}{(}\PY{p}{[}\PY{l+m+mi}{1}\PY{p}{,}\PY{l+m+mi}{2}\PY{p}{,}\PY{l+m+mi}{3}\PY{p}{]}\PY{p}{)}\PY{p}{,} \PY{n}{np}\PY{o}{.}\PY{n}{array}\PY{p}{(}\PY{p}{[}\PY{l+m+mi}{3}\PY{p}{,}\PY{l+m+mi}{4}\PY{p}{,}\PY{l+m+mi}{5}\PY{p}{]}\PY{p}{)}\PY{p}{)}\PY{p}{)}
\PY{n+nb}{print}\PY{p}{(}\PY{n}{vector\PYZus{}compare}\PY{p}{(}\PY{n}{np}\PY{o}{.}\PY{n}{array}\PY{p}{(}\PY{p}{[}\PY{l+m+mi}{1}\PY{p}{,}\PY{l+m+mi}{2}\PY{p}{,}\PY{l+m+mi}{3}\PY{p}{]}\PY{p}{)}\PY{p}{,} \PY{n}{np}\PY{o}{.}\PY{n}{array}\PY{p}{(}\PY{p}{[}\PY{l+m+mi}{4}\PY{p}{,}\PY{l+m+mi}{5}\PY{p}{,}\PY{l+m+mi}{6}\PY{p}{]}\PY{p}{)}\PY{p}{)}\PY{p}{)}
\PY{n+nb}{print}\PY{p}{(}\PY{n}{vector\PYZus{}compare}\PY{p}{(}\PY{n}{np}\PY{o}{.}\PY{n}{array}\PY{p}{(}\PY{p}{[}\PY{l+m+mi}{1}\PY{p}{,}\PY{l+m+mi}{2}\PY{p}{,}\PY{l+m+mi}{3}\PY{p}{]}\PY{p}{)}\PY{p}{,} \PY{n}{np}\PY{o}{.}\PY{n}{array}\PY{p}{(}\PY{p}{[}\PY{l+m+mi}{5}\PY{p}{,}\PY{l+m+mi}{6}\PY{p}{,}\PY{l+m+mi}{7}\PY{p}{]}\PY{p}{)}\PY{p}{)}\PY{p}{)}
\PY{n+nb}{print}\PY{p}{(}\PY{n}{vector\PYZus{}compare}\PY{p}{(}\PY{n}{np}\PY{o}{.}\PY{n}{array}\PY{p}{(}\PY{p}{[}\PY{l+m+mi}{1}\PY{p}{,}\PY{l+m+mi}{2}\PY{p}{,}\PY{l+m+mi}{3}\PY{p}{]}\PY{p}{)}\PY{p}{,} \PY{n}{np}\PY{o}{.}\PY{n}{array}\PY{p}{(}\PY{p}{[}\PY{l+m+mi}{6}\PY{p}{,}\PY{l+m+mi}{7}\PY{p}{,}\PY{l+m+mi}{8}\PY{p}{]}\PY{p}{)}\PY{p}{)}\PY{p}{)}
\end{Verbatim}
\end{tcolorbox}

    \hypertarget{part-e}{%
\subsubsection{Part (e)}\label{part-e}}

Run the following code that runs vector\_compare on every subsequence in
the song- it will probably take at least 5 minutes. How do you interpret
this plot to find where the clip is in the song?

    \begin{tcolorbox}[breakable, size=fbox, boxrule=1pt, pad at break*=1mm,colback=cellbackground, colframe=cellborder]
\prompt{In}{incolor}{ }{\hspace{4pt}}
\begin{Verbatim}[commandchars=\\\{\}]
\PY{k+kn}{import} \PY{n+nn}{time}

\PY{n}{t0} \PY{o}{=} \PY{n}{time}\PY{o}{.}\PY{n}{time}\PY{p}{(}\PY{p}{)}
\PY{n}{idxs}\PY{p}{,} \PY{n}{song\PYZus{}compare} \PY{o}{=} \PY{n}{run\PYZus{}comparison}\PY{p}{(}\PY{n}{target\PYZus{}signal}\PY{p}{,} \PY{n}{given\PYZus{}signal}\PY{p}{)}
\PY{n}{t1} \PY{o}{=} \PY{n}{time}\PY{o}{.}\PY{n}{time}\PY{p}{(}\PY{p}{)}
\PY{n}{plt}\PY{o}{.}\PY{n}{plot}\PY{p}{(}\PY{n}{idxs}\PY{p}{,} \PY{n}{song\PYZus{}compare}\PY{p}{)}
\PY{n+nb}{print} \PY{p}{(}\PY{l+s+s2}{\PYZdq{}}\PY{l+s+s2}{That took }\PY{l+s+si}{\PYZpc{}(time).2f}\PY{l+s+s2}{ minutes to run}\PY{l+s+s2}{\PYZdq{}} \PY{o}{\PYZpc{}} \PY{p}{\PYZob{}}\PY{l+s+s1}{\PYZsq{}}\PY{l+s+s1}{time}\PY{l+s+s1}{\PYZsq{}}\PY{p}{:}\PY{p}{(}\PY{n}{t1}\PY{o}{\PYZhy{}}\PY{n}{t0}\PY{p}{)}\PY{o}{/}\PY{l+m+mf}{60.0}\PY{p}{\PYZcb{}} \PY{p}{)}
\end{Verbatim}
\end{tcolorbox}

    \hypertarget{question-5-gps-receivers}{%
\subsection{Question 5: GPS Receivers}\label{question-5-gps-receivers}}

    \begin{tcolorbox}[breakable, size=fbox, boxrule=1pt, pad at break*=1mm,colback=cellbackground, colframe=cellborder]
\prompt{In}{incolor}{2}{\hspace{4pt}}
\begin{Verbatim}[commandchars=\\\{\}]
\PY{o}{\PYZpc{}}\PY{k}{pylab} inline
\PY{k+kn}{import} \PY{n+nn}{numpy} \PY{k}{as} \PY{n+nn}{np}
\PY{k+kn}{import} \PY{n+nn}{matplotlib}\PY{n+nn}{.}\PY{n+nn}{pyplot} \PY{k}{as} \PY{n+nn}{plt}
\PY{k+kn}{import} \PY{n+nn}{scipy}\PY{n+nn}{.}\PY{n+nn}{io}
\PY{k+kn}{import} \PY{n+nn}{sys}
\end{Verbatim}
\end{tcolorbox}

    \begin{Verbatim}[commandchars=\\\{\}]
Populating the interactive namespace from numpy and matplotlib
\end{Verbatim}

    \begin{tcolorbox}[breakable, size=fbox, boxrule=1pt, pad at break*=1mm,colback=cellbackground, colframe=cellborder]
\prompt{In}{incolor}{3}{\hspace{4pt}}
\begin{Verbatim}[commandchars=\\\{\}]
\PY{c+c1}{\PYZsh{}\PYZsh{} RUN THIS FUNCTION BEFORE YOU START THIS PROBLEM}
\PY{c+c1}{\PYZsh{}\PYZsh{} This function will generate the gold code associated with the satellite ID using linear shift registers}
\PY{c+c1}{\PYZsh{}\PYZsh{} The satellite\PYZus{}ID can be any integer between 1 and 24}
\PY{k}{def} \PY{n+nf}{Gold\PYZus{}code\PYZus{}satellite}\PY{p}{(}\PY{n}{satellite\PYZus{}ID}\PY{p}{)}\PY{p}{:}
    \PY{n}{codelength} \PY{o}{=} \PY{l+m+mi}{1023}
    \PY{n}{registerlength} \PY{o}{=} \PY{l+m+mi}{10}
    
    \PY{c+c1}{\PYZsh{} Defining the MLS for G1 generator}
    \PY{n}{register1} \PY{o}{=} \PY{o}{\PYZhy{}}\PY{l+m+mi}{1}\PY{o}{*}\PY{n}{np}\PY{o}{.}\PY{n}{ones}\PY{p}{(}\PY{n}{registerlength}\PY{p}{)}
    \PY{n}{MLS1} \PY{o}{=} \PY{n}{np}\PY{o}{.}\PY{n}{zeros}\PY{p}{(}\PY{n}{codelength}\PY{p}{)}
    \PY{k}{for} \PY{n}{i} \PY{o+ow}{in} \PY{n+nb}{range}\PY{p}{(}\PY{n}{codelength}\PY{p}{)}\PY{p}{:}
        \PY{n}{MLS1}\PY{p}{[}\PY{n}{i}\PY{p}{]} \PY{o}{=} \PY{n}{register1}\PY{p}{[}\PY{l+m+mi}{9}\PY{p}{]}
        \PY{n}{modulo} \PY{o}{=} \PY{n}{register1}\PY{p}{[}\PY{l+m+mi}{2}\PY{p}{]}\PY{o}{*}\PY{n}{register1}\PY{p}{[}\PY{l+m+mi}{9}\PY{p}{]}
        \PY{n}{register1} \PY{o}{=} \PY{n}{np}\PY{o}{.}\PY{n}{roll}\PY{p}{(}\PY{n}{register1}\PY{p}{,}\PY{l+m+mi}{1}\PY{p}{)}
        \PY{n}{register1}\PY{p}{[}\PY{l+m+mi}{0}\PY{p}{]} \PY{o}{=} \PY{n}{modulo}
    
    \PY{c+c1}{\PYZsh{} Defining the MLS for G2 generator}
    \PY{n}{register2} \PY{o}{=} \PY{o}{\PYZhy{}}\PY{l+m+mi}{1}\PY{o}{*}\PY{n}{np}\PY{o}{.}\PY{n}{ones}\PY{p}{(}\PY{n}{registerlength}\PY{p}{)}
    \PY{n}{MLS2} \PY{o}{=} \PY{n}{np}\PY{o}{.}\PY{n}{zeros}\PY{p}{(}\PY{n}{codelength}\PY{p}{)}
    \PY{k}{for} \PY{n}{j} \PY{o+ow}{in} \PY{n+nb}{range}\PY{p}{(}\PY{n}{codelength}\PY{p}{)}\PY{p}{:}
        \PY{n}{MLS2}\PY{p}{[}\PY{n}{j}\PY{p}{]} \PY{o}{=} \PY{n}{register2}\PY{p}{[}\PY{l+m+mi}{9}\PY{p}{]}
        \PY{n}{modulo} \PY{o}{=} \PY{n}{register2}\PY{p}{[}\PY{l+m+mi}{1}\PY{p}{]}\PY{o}{*}\PY{n}{register2}\PY{p}{[}\PY{l+m+mi}{2}\PY{p}{]}\PY{o}{*}\PY{n}{register2}\PY{p}{[}\PY{l+m+mi}{5}\PY{p}{]}\PY{o}{*}\PY{n}{register2}\PY{p}{[}\PY{l+m+mi}{7}\PY{p}{]}\PY{o}{*}\PY{n}{register2}\PY{p}{[}\PY{l+m+mi}{8}\PY{p}{]}\PY{o}{*}\PY{n}{register2}\PY{p}{[}\PY{l+m+mi}{9}\PY{p}{]}
        \PY{n}{register2} \PY{o}{=} \PY{n}{np}\PY{o}{.}\PY{n}{roll}\PY{p}{(}\PY{n}{register2}\PY{p}{,}\PY{l+m+mi}{1}\PY{p}{)}
        \PY{n}{register2}\PY{p}{[}\PY{l+m+mi}{0}\PY{p}{]} \PY{o}{=} \PY{n}{modulo}
    
    \PY{n}{delay} \PY{o}{=} \PY{n}{np}\PY{o}{.}\PY{n}{array}\PY{p}{(}\PY{p}{[}\PY{l+m+mi}{5}\PY{p}{,}\PY{l+m+mi}{6}\PY{p}{,}\PY{l+m+mi}{7}\PY{p}{,}\PY{l+m+mi}{8}\PY{p}{,}\PY{l+m+mi}{17}\PY{p}{,}\PY{l+m+mi}{18}\PY{p}{,}\PY{l+m+mi}{139}\PY{p}{,}\PY{l+m+mi}{140}\PY{p}{,}\PY{l+m+mi}{141}\PY{p}{,}\PY{l+m+mi}{251}\PY{p}{,}\PY{l+m+mi}{252}\PY{p}{,}\PY{l+m+mi}{254}\PY{p}{,}\PY{l+m+mi}{255}\PY{p}{,}\PY{l+m+mi}{256}\PY{p}{,}\PY{l+m+mi}{257}\PY{p}{,}\PY{l+m+mi}{258}\PY{p}{,}\PY{l+m+mi}{469}\PY{p}{,}\PY{l+m+mi}{470}\PY{p}{,}\PY{l+m+mi}{471}\PY{p}{,}\PY{l+m+mi}{472}\PY{p}{,}\PY{l+m+mi}{473}\PY{p}{,}\PY{l+m+mi}{474}\PY{p}{,}\PY{l+m+mi}{509}\PY{p}{,}\PY{l+m+mi}{512}\PY{p}{,}\PY{l+m+mi}{513}\PY{p}{,}\PY{l+m+mi}{514}\PY{p}{,}\PY{l+m+mi}{515}\PY{p}{,}\PY{l+m+mi}{516}\PY{p}{,}\PY{l+m+mi}{859}\PY{p}{,}\PY{l+m+mi}{860}\PY{p}{,}\PY{l+m+mi}{861}\PY{p}{,}\PY{l+m+mi}{862}\PY{p}{]}\PY{p}{)}
    \PY{n}{G1\PYZus{}out} \PY{o}{=} \PY{n}{MLS1}\PY{p}{;}
    \PY{n}{shamt} \PY{o}{=} \PY{n}{delay}\PY{p}{[}\PY{n}{satellite\PYZus{}ID} \PY{o}{\PYZhy{}} \PY{l+m+mi}{1}\PY{p}{]}
    \PY{n}{G2\PYZus{}out} \PY{o}{=} \PY{n}{np}\PY{o}{.}\PY{n}{roll}\PY{p}{(}\PY{n}{MLS2}\PY{p}{,}\PY{n}{shamt}\PY{p}{)}
    
    \PY{n}{CA\PYZus{}code} \PY{o}{=} \PY{n}{G1\PYZus{}out} \PY{o}{*} \PY{n}{G2\PYZus{}out}
    
    \PY{k}{return} \PY{n}{CA\PYZus{}code}
\end{Verbatim}
\end{tcolorbox}

    \hypertarget{part-a}{%
\subsubsection{Part (a)}\label{part-a}}

    \begin{tcolorbox}[breakable, size=fbox, boxrule=1pt, pad at break*=1mm,colback=cellbackground, colframe=cellborder]
\prompt{In}{incolor}{4}{\hspace{4pt}}
\begin{Verbatim}[commandchars=\\\{\}]
\PY{k}{def} \PY{n+nf}{cross\PYZus{}correlation}\PY{p}{(}\PY{n}{array1}\PY{p}{,} \PY{n}{array2}\PY{p}{)}\PY{p}{:}
    \PY{l+s+sd}{\PYZdq{}\PYZdq{}\PYZdq{} This function should return two arrays or a matrix with one row corresponding to }
\PY{l+s+sd}{    the offset and other to the correlation value. array1 and array2 do not have to be}
\PY{l+s+sd}{    arrays of equal length.}
\PY{l+s+sd}{    Think of array1 as the received signal and array2 as the signature.}
\PY{l+s+sd}{    The function should return correlation values as well as the indices of the nonzero values of the correlation}
\PY{l+s+sd}{    Hint: look up np.correlate}
\PY{l+s+sd}{    \PYZdq{}\PYZdq{}\PYZdq{}}

    \PY{c+c1}{\PYZsh{}correlated\PYZus{}array = \PYZsh{}Your code here (it is just one line) np.correlate(array1, array2, \PYZsq{}full\PYZsq{})}
    \PY{n}{correlated\PYZus{}array} \PY{o}{=} \PY{n}{np}\PY{o}{.}\PY{n}{correlate}\PY{p}{(}\PY{n}{array1}\PY{p}{,} \PY{n}{array2}\PY{p}{,} \PY{l+s+s1}{\PYZsq{}}\PY{l+s+s1}{full}\PY{l+s+s1}{\PYZsq{}}\PY{p}{)}
    
    \PY{c+c1}{\PYZsh{}Since both the arrays start at 0, the last \PYZdq{}shift\PYZdq{} where the signals overlap is the length of the first array}
    \PY{n}{end\PYZus{}index} \PY{o}{=} \PY{n+nb}{len}\PY{p}{(}\PY{n}{array1}\PY{p}{)}
    
    \PY{c+c1}{\PYZsh{}Similarly, the first \PYZdq{}shift\PYZdq{} where the signals overlap is the negative of the length of the second array, offset by one because of the zero index.}
    \PY{n}{st\PYZus{}index} \PY{o}{=} \PY{o}{\PYZhy{}}\PY{n+nb}{len}\PY{p}{(}\PY{n}{array2}\PY{p}{)} \PY{o}{+} \PY{l+m+mi}{1}
    
    \PY{n}{indices} \PY{o}{=} \PY{n}{np}\PY{o}{.}\PY{n}{arange}\PY{p}{(}\PY{n}{st\PYZus{}index}\PY{p}{,} \PY{n}{end\PYZus{}index}\PY{p}{)}    
    \PY{k}{return} \PY{p}{(}\PY{n}{indices}\PY{p}{,} \PY{n}{correlated\PYZus{}array}\PY{p}{)}
\end{Verbatim}
\end{tcolorbox}

    \begin{tcolorbox}[breakable, size=fbox, boxrule=1pt, pad at break*=1mm,colback=cellbackground, colframe=cellborder]
\prompt{In}{incolor}{5}{\hspace{4pt}}
\begin{Verbatim}[commandchars=\\\{\}]
\PY{c+c1}{\PYZsh{} Plot the auto\PYZhy{}correlation of satellite 10 with itself. Fill in the function call.}
\PY{n}{array\PYZus{}10} \PY{o}{=} \PY{n}{Gold\PYZus{}code\PYZus{}satellite}\PY{p}{(}\PY{l+m+mi}{10}\PY{p}{)}

\PY{p}{(}\PY{n}{ind\PYZus{}10}\PY{p}{,} \PY{n}{self\PYZus{}10}\PY{p}{)} \PY{o}{=} \PY{n}{cross\PYZus{}correlation}\PY{p}{(}\PY{n}{array\PYZus{}10}\PY{p}{,} \PY{n}{array\PYZus{}10}\PY{p}{)}

\PY{n}{plt}\PY{o}{.}\PY{n}{figure}\PY{p}{(}\PY{n}{figsize}\PY{o}{=}\PY{p}{(}\PY{l+m+mi}{16}\PY{p}{,} \PY{l+m+mi}{4}\PY{p}{)}\PY{p}{)}
\PY{n}{plt}\PY{o}{.}\PY{n}{stem}\PY{p}{(}\PY{n}{ind\PYZus{}10}\PY{p}{,} \PY{n}{self\PYZus{}10}\PY{p}{)}
\PY{n}{plt}\PY{o}{.}\PY{n}{xlabel}\PY{p}{(}\PY{l+s+s2}{\PYZdq{}}\PY{l+s+s2}{Index of correlation}\PY{l+s+s2}{\PYZdq{}}\PY{p}{)}
\PY{n}{plt}\PY{o}{.}\PY{n}{ylabel}\PY{p}{(}\PY{l+s+s2}{\PYZdq{}}\PY{l+s+s2}{Correlation value}\PY{l+s+s2}{\PYZdq{}}\PY{p}{)}
\end{Verbatim}
\end{tcolorbox}

            \begin{tcolorbox}[breakable, boxrule=.5pt, size=fbox, pad at break*=1mm, opacityfill=0]
\prompt{Out}{outcolor}{5}{\hspace{3.5pt}}
\begin{Verbatim}[commandchars=\\\{\}]
Text(0, 0.5, 'Correlation value')
\end{Verbatim}
\end{tcolorbox}
        
    \begin{center}
    \adjustimage{max size={0.9\linewidth}{0.9\paperheight}}{output_17_2.png}
    \end{center}
    { \hspace*{\fill} \\}
    
    The autocorrelation peaks at 1023 when the signals are perfectly aligned
(offset 0). The correlation of a Gold code with a shifted version of
itself is not significant.

    \hypertarget{part-b}{%
\subsubsection{Part (b)}\label{part-b}}

Plot the cross correlation when array1 = satellite 13 and array2 =
satellite10

    \begin{tcolorbox}[breakable, size=fbox, boxrule=1pt, pad at break*=1mm,colback=cellbackground, colframe=cellborder]
\prompt{In}{incolor}{6}{\hspace{4pt}}
\begin{Verbatim}[commandchars=\\\{\}]
\PY{c+c1}{\PYZsh{}Your code here}
\PY{n}{array\PYZus{}10} \PY{o}{=} \PY{n}{Gold\PYZus{}code\PYZus{}satellite}\PY{p}{(}\PY{l+m+mi}{10}\PY{p}{)}
\PY{n}{array\PYZus{}13} \PY{o}{=} \PY{n}{Gold\PYZus{}code\PYZus{}satellite}\PY{p}{(}\PY{l+m+mi}{13}\PY{p}{)}

\PY{p}{(}\PY{n}{ind\PYZus{}10}\PY{p}{,} \PY{n}{self\PYZus{}10}\PY{p}{)} \PY{o}{=} \PY{n}{cross\PYZus{}correlation}\PY{p}{(}\PY{n}{array\PYZus{}13}\PY{p}{,} \PY{n}{array\PYZus{}10}\PY{p}{)}

\PY{n}{plt}\PY{o}{.}\PY{n}{figure}\PY{p}{(}\PY{n}{figsize}\PY{o}{=}\PY{p}{(}\PY{l+m+mi}{16}\PY{p}{,} \PY{l+m+mi}{4}\PY{p}{)}\PY{p}{)}
\PY{n}{plt}\PY{o}{.}\PY{n}{stem}\PY{p}{(}\PY{n}{ind\PYZus{}10}\PY{p}{,} \PY{n}{self\PYZus{}10}\PY{p}{)}
\PY{n}{plt}\PY{o}{.}\PY{n}{xlabel}\PY{p}{(}\PY{l+s+s2}{\PYZdq{}}\PY{l+s+s2}{Index of correlation}\PY{l+s+s2}{\PYZdq{}}\PY{p}{)}
\PY{n}{plt}\PY{o}{.}\PY{n}{ylabel}\PY{p}{(}\PY{l+s+s2}{\PYZdq{}}\PY{l+s+s2}{Correlation value}\PY{l+s+s2}{\PYZdq{}}\PY{p}{)}
\end{Verbatim}
\end{tcolorbox}

            \begin{tcolorbox}[breakable, boxrule=.5pt, size=fbox, pad at break*=1mm, opacityfill=0]
\prompt{Out}{outcolor}{6}{\hspace{3.5pt}}
\begin{Verbatim}[commandchars=\\\{\}]
Text(0, 0.5, 'Correlation value')
\end{Verbatim}
\end{tcolorbox}
        
    \begin{center}
    \adjustimage{max size={0.9\linewidth}{0.9\paperheight}}{output_20_2.png}
    \end{center}
    { \hspace*{\fill} \\}
    
    We see that the cross-correlation of a Gold code of any satellite with
any other satellite is very low. This indicates that when given some
unknown data, we can differentiate between different satellites.

    \hypertarget{part-c}{%
\subsubsection{Part (c)}\label{part-c}}

    \begin{tcolorbox}[breakable, size=fbox, boxrule=1pt, pad at break*=1mm,colback=cellbackground, colframe=cellborder]
\prompt{In}{incolor}{7}{\hspace{4pt}}
\begin{Verbatim}[commandchars=\\\{\}]
\PY{c+c1}{\PYZsh{}\PYZsh{} THIS IS A HELPER FUNCTION FOR PART C THAT GENERATES +\PYZhy{}1 RANDOM NOISE}
\PY{k}{def} \PY{n+nf}{integernoise\PYZus{}generator}\PY{p}{(}\PY{n}{length\PYZus{}of\PYZus{}noise}\PY{p}{)}\PY{p}{:}
    \PY{n}{noise\PYZus{}array} \PY{o}{=} \PY{n}{np}\PY{o}{.}\PY{n}{random}\PY{o}{.}\PY{n}{randint}\PY{p}{(}\PY{l+m+mi}{2}\PY{p}{,} \PY{n}{size} \PY{o}{=} \PY{n}{length\PYZus{}of\PYZus{}noise}\PY{p}{)}
    \PY{n}{noise\PYZus{}array} \PY{o}{=} \PY{l+m+mi}{2} \PY{o}{*} \PY{n}{noise\PYZus{}array} \PY{o}{\PYZhy{}} \PY{n}{np}\PY{o}{.}\PY{n}{ones}\PY{p}{(}\PY{n}{size}\PY{p}{(}\PY{n}{noise\PYZus{}array}\PY{p}{)}\PY{p}{)}
    \PY{k}{return} \PY{n}{noise\PYZus{}array}

\PY{c+c1}{\PYZsh{} YOUR CODE HERE}

\PY{n}{noise} \PY{o}{=} \PY{n}{integernoise\PYZus{}generator}\PY{p}{(}\PY{l+m+mi}{1023}\PY{p}{)}

\PY{p}{(}\PY{n}{ind\PYZus{}10}\PY{p}{,} \PY{n}{self\PYZus{}10}\PY{p}{)} \PY{o}{=} \PY{n}{cross\PYZus{}correlation}\PY{p}{(}\PY{n}{noise}\PY{p}{,} \PY{n}{array\PYZus{}10}\PY{p}{)}

\PY{n}{plt}\PY{o}{.}\PY{n}{figure}\PY{p}{(}\PY{n}{figsize}\PY{o}{=}\PY{p}{(}\PY{l+m+mi}{16}\PY{p}{,} \PY{l+m+mi}{4}\PY{p}{)}\PY{p}{)}
\PY{n}{plt}\PY{o}{.}\PY{n}{stem}\PY{p}{(}\PY{n}{ind\PYZus{}10}\PY{p}{,} \PY{n}{self\PYZus{}10}\PY{p}{)}
\PY{n}{plt}\PY{o}{.}\PY{n}{xlabel}\PY{p}{(}\PY{l+s+s2}{\PYZdq{}}\PY{l+s+s2}{Index of correlation}\PY{l+s+s2}{\PYZdq{}}\PY{p}{)}
\PY{n}{plt}\PY{o}{.}\PY{n}{ylabel}\PY{p}{(}\PY{l+s+s2}{\PYZdq{}}\PY{l+s+s2}{Correlation value}\PY{l+s+s2}{\PYZdq{}}\PY{p}{)}
\end{Verbatim}
\end{tcolorbox}

            \begin{tcolorbox}[breakable, boxrule=.5pt, size=fbox, pad at break*=1mm, opacityfill=0]
\prompt{Out}{outcolor}{7}{\hspace{3.5pt}}
\begin{Verbatim}[commandchars=\\\{\}]
Text(0, 0.5, 'Correlation value')
\end{Verbatim}
\end{tcolorbox}
        
    \begin{center}
    \adjustimage{max size={0.9\linewidth}{0.9\paperheight}}{output_23_2.png}
    \end{center}
    { \hspace*{\fill} \\}
    
    We see that the cross-correlation of the Gold code of any satellite with
integer noise is very low. This indicates that we can still figure out
the presence of a satellite even if it is buried in noise.

    \hypertarget{part-d}{%
\subsubsection{Part (d)}\label{part-d}}

    \begin{tcolorbox}[breakable, size=fbox, boxrule=1pt, pad at break*=1mm,colback=cellbackground, colframe=cellborder]
\prompt{In}{incolor}{8}{\hspace{4pt}}
\begin{Verbatim}[commandchars=\\\{\}]
\PY{c+c1}{\PYZsh{}\PYZsh{} THIS IS A HELPER FUNCTION FOR PART D THAT GENERATES REAL VALUED RANDOM NOISE}
\PY{k}{def} \PY{n+nf}{gaussiannoise\PYZus{}generator}\PY{p}{(}\PY{n}{length\PYZus{}of\PYZus{}noise}\PY{p}{)}\PY{p}{:}
    \PY{n}{noise\PYZus{}array} \PY{o}{=} \PY{n}{np}\PY{o}{.}\PY{n}{random}\PY{o}{.}\PY{n}{normal}\PY{p}{(}\PY{l+m+mi}{0}\PY{p}{,} \PY{l+m+mi}{1}\PY{p}{,} \PY{n}{length\PYZus{}of\PYZus{}noise}\PY{p}{)}
    \PY{k}{return} \PY{n}{noise\PYZus{}array}

\PY{c+c1}{\PYZsh{} YOUR CODE HERE}

\PY{n}{gauss\PYZus{}noise} \PY{o}{=} \PY{n}{gaussiannoise\PYZus{}generator}\PY{p}{(}\PY{l+m+mi}{1023}\PY{p}{)}

\PY{p}{(}\PY{n}{ind\PYZus{}10}\PY{p}{,} \PY{n}{self\PYZus{}10}\PY{p}{)} \PY{o}{=} \PY{n}{cross\PYZus{}correlation}\PY{p}{(}\PY{n}{gauss\PYZus{}noise}\PY{p}{,} \PY{n}{array\PYZus{}10}\PY{p}{)}

\PY{n}{plt}\PY{o}{.}\PY{n}{figure}\PY{p}{(}\PY{n}{figsize}\PY{o}{=}\PY{p}{(}\PY{l+m+mi}{16}\PY{p}{,} \PY{l+m+mi}{4}\PY{p}{)}\PY{p}{)}
\PY{n}{plt}\PY{o}{.}\PY{n}{stem}\PY{p}{(}\PY{n}{ind\PYZus{}10}\PY{p}{,} \PY{n}{self\PYZus{}10}\PY{p}{)}
\PY{n}{plt}\PY{o}{.}\PY{n}{xlabel}\PY{p}{(}\PY{l+s+s2}{\PYZdq{}}\PY{l+s+s2}{Index of correlation}\PY{l+s+s2}{\PYZdq{}}\PY{p}{)}
\PY{n}{plt}\PY{o}{.}\PY{n}{ylabel}\PY{p}{(}\PY{l+s+s2}{\PYZdq{}}\PY{l+s+s2}{Correlation value}\PY{l+s+s2}{\PYZdq{}}\PY{p}{)}
\end{Verbatim}
\end{tcolorbox}

            \begin{tcolorbox}[breakable, boxrule=.5pt, size=fbox, pad at break*=1mm, opacityfill=0]
\prompt{Out}{outcolor}{8}{\hspace{3.5pt}}
\begin{Verbatim}[commandchars=\\\{\}]
Text(0, 0.5, 'Correlation value')
\end{Verbatim}
\end{tcolorbox}
        
    \begin{center}
    \adjustimage{max size={0.9\linewidth}{0.9\paperheight}}{output_26_2.png}
    \end{center}
    { \hspace*{\fill} \\}
    
    We see that the Gold code of any satellite with Gaussian noise is very
low. This indicates that we can still figure out the presence of a
satellite even if it is buried in Gaussian noise.

    \hypertarget{part-e}{%
\subsubsection{Part (e)}\label{part-e}}

Hint: you can use a absolute value threshold of 800 for the
cross-correlation to detect if a given satellite is present. np.argwhere
may be useful for detecting peak locations.

    \begin{tcolorbox}[breakable, size=fbox, boxrule=1pt, pad at break*=1mm,colback=cellbackground, colframe=cellborder]
\prompt{In}{incolor}{9}{\hspace{4pt}}
\begin{Verbatim}[commandchars=\\\{\}]
\PY{c+c1}{\PYZsh{}Now let us see which signals are present in the data signal that is in data1.npy}
\PY{n}{signal1} \PY{o}{=} \PY{n}{np}\PY{o}{.}\PY{n}{load}\PY{p}{(}\PY{l+s+s1}{\PYZsq{}}\PY{l+s+s1}{data1.npy}\PY{l+s+s1}{\PYZsq{}}\PY{p}{)}
\end{Verbatim}
\end{tcolorbox}

    \begin{tcolorbox}[breakable, size=fbox, boxrule=1pt, pad at break*=1mm,colback=cellbackground, colframe=cellborder]
\prompt{In}{incolor}{10}{\hspace{4pt}}
\begin{Verbatim}[commandchars=\\\{\}]
\PY{c+c1}{\PYZsh{}Here try plotting the cross\PYZhy{}correlations of data1.npy with a few of the satellite gold codes. }
\PY{c+c1}{\PYZsh{}How can you detect if the satellite is present?}

\PY{p}{(}\PY{n}{ind\PYZus{}10}\PY{p}{,} \PY{n}{self\PYZus{}10}\PY{p}{)} \PY{o}{=} \PY{n}{cross\PYZus{}correlation}\PY{p}{(}\PY{n}{signal1}\PY{p}{,} \PY{n}{array\PYZus{}13}\PY{p}{)}

\PY{n}{plt}\PY{o}{.}\PY{n}{figure}\PY{p}{(}\PY{n}{figsize}\PY{o}{=}\PY{p}{(}\PY{l+m+mi}{16}\PY{p}{,} \PY{l+m+mi}{4}\PY{p}{)}\PY{p}{)}
\PY{n}{plt}\PY{o}{.}\PY{n}{stem}\PY{p}{(}\PY{n}{ind\PYZus{}10}\PY{p}{,} \PY{n}{self\PYZus{}10}\PY{p}{)}
\PY{n}{plt}\PY{o}{.}\PY{n}{xlabel}\PY{p}{(}\PY{l+s+s2}{\PYZdq{}}\PY{l+s+s2}{Index of correlation}\PY{l+s+s2}{\PYZdq{}}\PY{p}{)}
\PY{n}{plt}\PY{o}{.}\PY{n}{ylabel}\PY{p}{(}\PY{l+s+s2}{\PYZdq{}}\PY{l+s+s2}{Correlation value}\PY{l+s+s2}{\PYZdq{}}\PY{p}{)}
\end{Verbatim}
\end{tcolorbox}

            \begin{tcolorbox}[breakable, boxrule=.5pt, size=fbox, pad at break*=1mm, opacityfill=0]
\prompt{Out}{outcolor}{10}{\hspace{3.5pt}}
\begin{Verbatim}[commandchars=\\\{\}]
Text(0, 0.5, 'Correlation value')
\end{Verbatim}
\end{tcolorbox}
        
    \begin{center}
    \adjustimage{max size={0.9\linewidth}{0.9\paperheight}}{output_30_2.png}
    \end{center}
    { \hspace*{\fill} \\}
    
    \begin{tcolorbox}[breakable, size=fbox, boxrule=1pt, pad at break*=1mm,colback=cellbackground, colframe=cellborder]
\prompt{In}{incolor}{11}{\hspace{4pt}}
\begin{Verbatim}[commandchars=\\\{\}]
\PY{c+c1}{\PYZsh{}\PYZsh{} This helper function returns 1 if peak (greater than threshold or less than \PYZhy{}threshold) is found, else it returns 0.}
\PY{c+c1}{\PYZsh{}\PYZsh{} You do not have to use this function as there are other solutions to this part as well}

\PY{k}{def} \PY{n+nf}{find\PYZus{}peak}\PY{p}{(}\PY{n}{signal}\PY{p}{,} \PY{n}{threshold}\PY{p}{)}\PY{p}{:}
    \PY{n}{max\PYZus{}value} \PY{o}{=} \PY{n}{np}\PY{o}{.}\PY{n}{amax}\PY{p}{(}\PY{n}{signal}\PY{p}{)}
    \PY{n}{min\PYZus{}value} \PY{o}{=} \PY{n}{np}\PY{o}{.}\PY{n}{amin}\PY{p}{(}\PY{n}{signal}\PY{p}{)}
    \PY{k}{if} \PY{n}{max\PYZus{}value} \PY{o}{\PYZgt{}} \PY{n}{threshold}\PY{p}{:}
        \PY{n}{ret\PYZus{}value} \PY{o}{=} \PY{l+m+mi}{1}
    \PY{k}{elif} \PY{n}{min\PYZus{}value} \PY{o}{\PYZlt{}} \PY{o}{\PYZhy{}}\PY{l+m+mi}{1} \PY{o}{*} \PY{n}{threshold}\PY{p}{:}
        \PY{n}{ret\PYZus{}value} \PY{o}{=} \PY{l+m+mi}{1}
    \PY{k}{else}\PY{p}{:}
        \PY{n}{ret\PYZus{}value} \PY{o}{=} \PY{l+m+mi}{0}
    \PY{k}{return} \PY{n}{ret\PYZus{}value}
\end{Verbatim}
\end{tcolorbox}

    \begin{tcolorbox}[breakable, size=fbox, boxrule=1pt, pad at break*=1mm,colback=cellbackground, colframe=cellborder]
\prompt{In}{incolor}{12}{\hspace{4pt}}
\begin{Verbatim}[commandchars=\\\{\}]
\PY{c+c1}{\PYZsh{}\PYZsh{} USE \PYZsq{}np.load\PYZsq{} FUNCTION TO LOAD THE DATA}
\PY{c+c1}{\PYZsh{}\PYZsh{} USE DATA1.NPY AS THE SIGNAL ARRAY}
\PY{c+c1}{\PYZsh{} YOUR CODE HERE}
\PY{n}{signal1} \PY{o}{=} \PY{n}{np}\PY{o}{.}\PY{n}{load}\PY{p}{(}\PY{l+s+s1}{\PYZsq{}}\PY{l+s+s1}{data1.npy}\PY{l+s+s1}{\PYZsq{}}\PY{p}{)}

\PY{k}{def} \PY{n+nf}{find\PYZus{}sat}\PY{p}{(}\PY{n}{signal}\PY{p}{)}\PY{p}{:}
    \PY{n}{sat\PYZus{}list} \PY{o}{=} \PY{p}{[}\PY{n}{cross\PYZus{}correlation}\PY{p}{(}\PY{n}{signal}\PY{p}{,} \PY{n}{Gold\PYZus{}code\PYZus{}satellite}\PY{p}{(}\PY{n}{i}\PY{p}{)}\PY{p}{)} \PY{k}{for} \PY{n}{i} \PY{o+ow}{in} \PY{n+nb}{range}\PY{p}{(}\PY{l+m+mi}{1}\PY{p}{,} \PY{l+m+mi}{25}\PY{p}{)}\PY{p}{]}
    \PY{n}{peaks} \PY{o}{=} \PY{p}{[}\PY{n}{j} \PY{o}{+} \PY{l+m+mi}{1} \PY{k}{for} \PY{n}{j} \PY{o+ow}{in} \PY{n+nb}{range}\PY{p}{(}\PY{n+nb}{len}\PY{p}{(}\PY{n}{sat\PYZus{}list}\PY{p}{)}\PY{p}{)} \PY{k}{if} \PY{n}{find\PYZus{}peak}\PY{p}{(}\PY{n}{sat\PYZus{}list}\PY{p}{[}\PY{n}{j}\PY{p}{]}\PY{p}{[}\PY{l+m+mi}{1}\PY{p}{]}\PY{p}{,} \PY{l+m+mi}{800}\PY{p}{)}\PY{p}{]}
    \PY{k}{return} \PY{n}{peaks}

\PY{n}{find\PYZus{}sat}\PY{p}{(}\PY{n}{signal1}\PY{p}{)}
\end{Verbatim}
\end{tcolorbox}

            \begin{tcolorbox}[breakable, boxrule=.5pt, size=fbox, pad at break*=1mm, opacityfill=0]
\prompt{Out}{outcolor}{12}{\hspace{3.5pt}}
\begin{Verbatim}[commandchars=\\\{\}]
[4, 7, 13, 19]
\end{Verbatim}
\end{tcolorbox}
        
    \hypertarget{part-f}{%
\subsubsection{Part (f)}\label{part-f}}

    \begin{tcolorbox}[breakable, size=fbox, boxrule=1pt, pad at break*=1mm,colback=cellbackground, colframe=cellborder]
\prompt{In}{incolor}{13}{\hspace{4pt}}
\begin{Verbatim}[commandchars=\\\{\}]
\PY{c+c1}{\PYZsh{}\PYZsh{} USE DATA2.NPY AS THE SIGNAL ARRAY}

\PY{c+c1}{\PYZsh{} YOUR CODE HERE \PYZhy{}\PYZhy{}\PYZhy{} first write code to figure out which satellite is present}
\PY{n}{signal2} \PY{o}{=} \PY{n}{np}\PY{o}{.}\PY{n}{load}\PY{p}{(}\PY{l+s+s1}{\PYZsq{}}\PY{l+s+s1}{data2.npy}\PY{l+s+s1}{\PYZsq{}}\PY{p}{)}

\PY{n}{sats} \PY{o}{=} \PY{n}{find\PYZus{}sat}\PY{p}{(}\PY{n}{signal2}\PY{p}{)}
\PY{n+nb}{print}\PY{p}{(}\PY{n}{sats}\PY{p}{)}
\end{Verbatim}
\end{tcolorbox}

    \begin{Verbatim}[commandchars=\\\{\}]
[3]
\end{Verbatim}

    \begin{tcolorbox}[breakable, size=fbox, boxrule=1pt, pad at break*=1mm,colback=cellbackground, colframe=cellborder]
\prompt{In}{incolor}{14}{\hspace{4pt}}
\begin{Verbatim}[commandchars=\\\{\}]
\PY{c+c1}{\PYZsh{}\PYZsh{} Once you have figured out which satellite is present, proceed to find the data transmitted}

\PY{k}{def} \PY{n+nf}{find\PYZus{}message}\PY{p}{(}\PY{n}{signal}\PY{p}{,} \PY{n}{sat\PYZus{}id}\PY{p}{)}\PY{p}{:}
    \PY{n}{message} \PY{o}{=} \PY{p}{[}\PY{p}{]}
    \PY{n}{mod\PYZus{}signal} \PY{o}{=} \PY{n}{cross\PYZus{}correlation}\PY{p}{(}\PY{n}{signal}\PY{p}{,} \PY{n}{Gold\PYZus{}code\PYZus{}satellite}\PY{p}{(}\PY{n}{sat\PYZus{}id}\PY{p}{)}\PY{p}{)}
    \PY{k}{for} \PY{n}{i} \PY{o+ow}{in} \PY{n}{mod\PYZus{}signal}\PY{p}{[}\PY{l+m+mi}{1}\PY{p}{]}\PY{p}{:}
        \PY{k}{if} \PY{n}{i} \PY{o}{\PYZgt{}} \PY{l+m+mi}{800}\PY{p}{:}
            \PY{n}{message} \PY{o}{+}\PY{o}{=} \PY{p}{[}\PY{l+m+mi}{1}\PY{p}{]}
        \PY{k}{elif} \PY{n}{i} \PY{o}{\PYZlt{}} \PY{o}{\PYZhy{}}\PY{l+m+mi}{800}\PY{p}{:}
            \PY{n}{message} \PY{o}{+}\PY{o}{=} \PY{p}{[}\PY{o}{\PYZhy{}}\PY{l+m+mi}{1}\PY{p}{]}
    \PY{k}{return} \PY{n}{message}

\PY{n+nb}{print}\PY{p}{(}\PY{p}{[}\PY{n}{find\PYZus{}message}\PY{p}{(}\PY{n}{signal2}\PY{p}{,} \PY{n}{i}\PY{p}{)} \PY{k}{for} \PY{n}{i} \PY{o+ow}{in} \PY{n}{sats}\PY{p}{]}\PY{p}{)}
\end{Verbatim}
\end{tcolorbox}

    \begin{Verbatim}[commandchars=\\\{\}]
[[1, -1, -1, -1, 1]]
\end{Verbatim}

    \hypertarget{part-g}{%
\subsubsection{Part (g)}\label{part-g}}

    \begin{tcolorbox}[breakable, size=fbox, boxrule=1pt, pad at break*=1mm,colback=cellbackground, colframe=cellborder]
\prompt{In}{incolor}{15}{\hspace{4pt}}
\begin{Verbatim}[commandchars=\\\{\}]
\PY{c+c1}{\PYZsh{}\PYZsh{} USE DATA3.NPY AS THE SIGNAL ARRAY}

\PY{c+c1}{\PYZsh{} YOUR CODE HERE}
\PY{n}{signal3} \PY{o}{=} \PY{n}{np}\PY{o}{.}\PY{n}{load}\PY{p}{(}\PY{l+s+s1}{\PYZsq{}}\PY{l+s+s1}{data3.npy}\PY{l+s+s1}{\PYZsq{}}\PY{p}{)}

\PY{n}{s3\PYZus{}sats} \PY{o}{=} \PY{n}{find\PYZus{}sat}\PY{p}{(}\PY{n}{signal3}\PY{p}{)}
\PY{n+nb}{print}\PY{p}{(}\PY{n}{s3\PYZus{}sats}\PY{p}{,} \PY{p}{[}\PY{n}{find\PYZus{}message}\PY{p}{(}\PY{n}{signal3}\PY{p}{,} \PY{n}{i}\PY{p}{)} \PY{k}{for} \PY{n}{i} \PY{o+ow}{in} \PY{n}{s3\PYZus{}sats}\PY{p}{]}\PY{p}{)}
\end{Verbatim}
\end{tcolorbox}

    \begin{Verbatim}[commandchars=\\\{\}]
[5, 20] [[1, 1, -1, -1, -1], [1, 1, -1, -1, -1]]
\end{Verbatim}

    \begin{tcolorbox}[breakable, size=fbox, boxrule=1pt, pad at break*=1mm,colback=cellbackground, colframe=cellborder]
\prompt{In}{incolor}{16}{\hspace{4pt}}
\begin{Verbatim}[commandchars=\\\{\}]
\PY{c+c1}{\PYZsh{}\PYZsh{} We know that the data is 1, 1, \PYZhy{}1, \PYZhy{}1, \PYZhy{}1, so we just find the positions of the first 1 in both the satellite correlations.}
\PY{c+c1}{\PYZsh{}\PYZsh{} plot the appropriate cross\PYZus{}correlation and find the location of the first 1}
\PY{c+c1}{\PYZsh{}\PYZsh{} Do this for as many satellites as there are present}
\PY{c+c1}{\PYZsh{}\PYZsh{} Your code here}

\PY{p}{(}\PY{n}{ind\PYZus{}5}\PY{p}{,} \PY{n}{self\PYZus{}5}\PY{p}{)} \PY{o}{=} \PY{n}{cross\PYZus{}correlation}\PY{p}{(}\PY{n}{signal3}\PY{p}{,} \PY{n}{Gold\PYZus{}code\PYZus{}satellite}\PY{p}{(}\PY{n}{s3\PYZus{}sats}\PY{p}{[}\PY{l+m+mi}{0}\PY{p}{]}\PY{p}{)}\PY{p}{)}

\PY{n}{plt}\PY{o}{.}\PY{n}{figure}\PY{p}{(}\PY{n}{figsize}\PY{o}{=}\PY{p}{(}\PY{l+m+mi}{16}\PY{p}{,} \PY{l+m+mi}{4}\PY{p}{)}\PY{p}{)}
\PY{n}{plt}\PY{o}{.}\PY{n}{stem}\PY{p}{(}\PY{n}{ind\PYZus{}5}\PY{p}{,} \PY{n}{self\PYZus{}5}\PY{p}{)}
\PY{n}{plt}\PY{o}{.}\PY{n}{xlabel}\PY{p}{(}\PY{l+s+s2}{\PYZdq{}}\PY{l+s+s2}{Index of correlation}\PY{l+s+s2}{\PYZdq{}}\PY{p}{)}
\PY{n}{plt}\PY{o}{.}\PY{n}{ylabel}\PY{p}{(}\PY{l+s+s2}{\PYZdq{}}\PY{l+s+s2}{Correlation value}\PY{l+s+s2}{\PYZdq{}}\PY{p}{)}

\PY{p}{(}\PY{n}{ind\PYZus{}20}\PY{p}{,} \PY{n}{self\PYZus{}20}\PY{p}{)} \PY{o}{=} \PY{n}{cross\PYZus{}correlation}\PY{p}{(}\PY{n}{signal3}\PY{p}{,} \PY{n}{Gold\PYZus{}code\PYZus{}satellite}\PY{p}{(}\PY{n}{s3\PYZus{}sats}\PY{p}{[}\PY{l+m+mi}{1}\PY{p}{]}\PY{p}{)}\PY{p}{)}

\PY{n}{plt}\PY{o}{.}\PY{n}{figure}\PY{p}{(}\PY{n}{figsize}\PY{o}{=}\PY{p}{(}\PY{l+m+mi}{16}\PY{p}{,} \PY{l+m+mi}{4}\PY{p}{)}\PY{p}{)}
\PY{n}{plt}\PY{o}{.}\PY{n}{stem}\PY{p}{(}\PY{n}{ind\PYZus{}20}\PY{p}{,} \PY{n}{self\PYZus{}20}\PY{p}{)}
\PY{n}{plt}\PY{o}{.}\PY{n}{xlabel}\PY{p}{(}\PY{l+s+s2}{\PYZdq{}}\PY{l+s+s2}{Index of correlation}\PY{l+s+s2}{\PYZdq{}}\PY{p}{)}
\PY{n}{plt}\PY{o}{.}\PY{n}{ylabel}\PY{p}{(}\PY{l+s+s2}{\PYZdq{}}\PY{l+s+s2}{Correlation value}\PY{l+s+s2}{\PYZdq{}}\PY{p}{)}
\end{Verbatim}
\end{tcolorbox}

            \begin{tcolorbox}[breakable, boxrule=.5pt, size=fbox, pad at break*=1mm, opacityfill=0]
\prompt{Out}{outcolor}{16}{\hspace{3.5pt}}
\begin{Verbatim}[commandchars=\\\{\}]
Text(0, 0.5, 'Correlation value')
\end{Verbatim}
\end{tcolorbox}
        
    \begin{center}
    \adjustimage{max size={0.9\linewidth}{0.9\paperheight}}{output_38_2.png}
    \end{center}
    { \hspace*{\fill} \\}
    
    \begin{center}
    \adjustimage{max size={0.9\linewidth}{0.9\paperheight}}{output_38_3.png}
    \end{center}
    { \hspace*{\fill} \\}
    
    \begin{tcolorbox}[breakable, size=fbox, boxrule=1pt, pad at break*=1mm,colback=cellbackground, colframe=cellborder]
\prompt{In}{incolor}{17}{\hspace{4pt}}
\begin{Verbatim}[commandchars=\\\{\}]
\PY{k}{def} \PY{n+nf}{find\PYZus{}offset}\PY{p}{(}\PY{n}{signal}\PY{p}{,} \PY{n}{sat\PYZus{}id}\PY{p}{)}\PY{p}{:}
    \PY{n}{corr} \PY{o}{=} \PY{n}{cross\PYZus{}correlation}\PY{p}{(}\PY{n}{signal}\PY{p}{,} \PY{n}{Gold\PYZus{}code\PYZus{}satellite}\PY{p}{(}\PY{n}{sat\PYZus{}id}\PY{p}{)}\PY{p}{)}
    \PY{k}{for} \PY{n}{i} \PY{o+ow}{in} \PY{n+nb}{range}\PY{p}{(}\PY{n+nb}{len}\PY{p}{(}\PY{n}{corr}\PY{p}{[}\PY{l+m+mi}{1}\PY{p}{]}\PY{p}{)}\PY{p}{)}\PY{p}{:}
        \PY{k}{if} \PY{n}{find\PYZus{}peak}\PY{p}{(}\PY{n}{corr}\PY{p}{[}\PY{l+m+mi}{1}\PY{p}{]}\PY{p}{[}\PY{n}{i}\PY{p}{]}\PY{p}{,} \PY{l+m+mi}{800}\PY{p}{)}\PY{p}{:}
            \PY{k}{return} \PY{n}{corr}\PY{p}{[}\PY{l+m+mi}{0}\PY{p}{]}\PY{p}{[}\PY{n}{i}\PY{p}{]}

\PY{n+nb}{print}\PY{p}{(}\PY{n}{find\PYZus{}offset}\PY{p}{(}\PY{n}{signal3}\PY{p}{,} \PY{l+m+mi}{5}\PY{p}{)}\PY{p}{,} \PY{n}{find\PYZus{}offset}\PY{p}{(}\PY{n}{signal3}\PY{p}{,} \PY{l+m+mi}{20}\PY{p}{)}\PY{p}{)}
\end{Verbatim}
\end{tcolorbox}

    \begin{Verbatim}[commandchars=\\\{\}]
0 506
\end{Verbatim}


    % Add a bibliography block to the postdoc
    
    
    
    \end{document}
