\documentclass[]{article}
\usepackage{amsmath}
\usepackage{amsfonts}
\usepackage{amssymb}
\usepackage{amsthm}
\usepackage{cancel}
\usepackage{graphicx}
\usepackage{pdfpages}
\usepackage{hyperref}

\renewcommand{\thesection}{\arabic{section}}
\renewcommand{\thesubsection}{\thesection.\alph{subsection}}
\renewcommand{\thesubsubsection}{\thesubsection.\roman{subsubsection}}

\newtheorem{genthm}{Theorem}

%opening
\title{EECS 16A HW06}
\author{Bryan Ngo}
\date{2019-10-12}

\begin{document}

\maketitle

\section{Elementary Matrices}

\subsection{}

\subsubsection{}

\begin{equation}
	\mathbf{E}_1 = \begin{bmatrix}
	0 & 0 & 1 & 0 \\
	0 & 1 & 0 & 0 \\
	1 & 0 & 0 & 0 \\
	0 & 0 & 0 & 1
	\end{bmatrix}
\end{equation}

\subsubsection{}

\begin{equation}
	\mathbf{E}_2 = \begin{bmatrix}
	1 & 0 & 0 & 0 \\
	0 & 1 & 0 & 0 \\
	0 & 0 & -5 & 0 \\
	0 & 0 & 0 & 1
	\end{bmatrix}
\end{equation}

\subsubsection{}

\begin{equation}
	\mathbf{E}_3 = \begin{bmatrix}
	1 & 0 & 0 & 0 \\
	0 & 1 & 0 & 0 \\
	0 & 0 & 1 & 0 \\
	0 & 3 & 0 & 1
	\end{bmatrix}
\end{equation}

\subsubsection{}

\begin{equation}
	\mathbf{E}_4 = \begin{bmatrix}
	1 & -1 & 0 & 0 \\
	0 & 1 & 0 & 0 \\
	0 & 0 & 1 & 0 \\
	0 & 0 & 0 & 1
	\end{bmatrix}
\end{equation}
The composite of the above matrix followed by the previous problem is 
\begin{equation}
	\mathbf{E}_3 \mathbf{E}_4 = 
	\begin{bmatrix}
	1 & 0 & 0 & 0 \\
	0 & 1 & 0 & 0 \\
	0 & 0 & 1 & 0 \\
	0 & 3 & 0 & 1
	\end{bmatrix}
	\begin{bmatrix}
	1 & -1 & 0 & 0 \\
	0 & 1 & 0 & 0 \\
	0 & 0 & 1 & 0 \\
	0 & 0 & 0 & 1
	\end{bmatrix}
	=
	\begin{bmatrix}
	1 & -1 & 0 & 0 \\
	0 & 1 & 0 & 0 \\
	0 & 0 & 1 & 0 \\
	0 & 3 & 0 & 1
	\end{bmatrix}
\end{equation}

\subsection{}

\(\mathbf{E}_1\) makes sense as the negative of \(r_1\) is moved down to \(r_4\). 
\(\mathbf{E}_2\) makes sense as \(r_2\) is multiplied by \(2\) and its negative is moved down to \(r_3\). 
\(\mathbf{E}_3\) makes sense as \(r_3\) is moved up to \(r_4\) and all of \(r_3\) is multiplied by \(-1\). 
Finally, \(\mathbf{E}_4\) makes sense as \(r_4\) is multiplied by 6 and its negative is moved up to \(r_3\), then \(-5r_4\) is moved up to \(r_1\), before \(r_4\) is multiplied by \(-1\). \\
\\
Verifying the matrix multiplication, 
\begin{align}
	\mathbf{E}_4 \mathbf{E}_3 \mathbf{E}_2 \mathbf{E}_1 &= 
	\left(\begin{bmatrix}
	1 & 0 & 0 & -5 \\
	0 & 1 & 0 & 0 \\
	0 & 0 & 1 & -6 \\
	0 & 0 & 0 & -1
	\end{bmatrix}
	\begin{bmatrix}
	1 & 0 & 0 & 0 \\
	0 & 1 & 0 & 0 \\
	0 & 0 & -1 & 0 \\
	0 & 0 & 1 & 1
	\end{bmatrix}\right)
	\left(\begin{bmatrix}
	1 & 0 & 0 & 0 \\
	0 & 1 & 0 & 0 \\
	0 & -2 & 1 & 0 \\
	0 & 0 & 0 & 1
	\end{bmatrix}
	\begin{bmatrix}
	1 & 0 & 0 & 0 \\
	0 & 1 & 0 & 0 \\
	0 & 0 & 1 & 0 \\
	-1 & 0 & 0 & 1
	\end{bmatrix}\right) \\
	&= 
	\begin{bmatrix}
	1 & 0 & -5 & -5 \\
	0 & 1 & 0 & 0 \\
	0 & 0 & -7 & -6 \\
	0 & 0 & -1 & -1
	\end{bmatrix}
	\begin{bmatrix}
	1 & 0 & 0 & 0 \\
	0 & 1 & 0 & 0 \\
	0 & -2 & 1 & 0 \\
	-1 & 0 & 0 & 1
	\end{bmatrix} \\
	&=
	\begin{bmatrix}
	6 & 10 & -5 & -5 \\
	0 & 1 & 0 & 0 \\
	6 & 14 & -7 & -6 \\
	1 & 2 & -1 & -1 \\
	\end{bmatrix}
\end{align}
Multiplying \(\mathbf{EA}\), 
\begin{align}
	\mathbf{EA} &= 
	\begin{bmatrix}
	6 & 10 & -5 & -5 \\
	0 & 1 & 0 & 0 \\
	6 & 14 & -7 & -6 \\
	1 & 2 & -1 & -1 \\
	\end{bmatrix}
	\left[\begin{array}{@{}cccc|c@{}}
	1 & 0 & 0 & -5 & 15 \\
	0 & 1 & 0 & 0 & -7 \\
	0 & 2 & -1 & 6 & 3 \\
	1 & 0 & 1 & -12 & -5
	\end{array}\right]
	&=
	\left[\begin{array}{@{}cccc|c@{}}
	1 & 0 & 0 & 0 & 30 \\
	0 & 1 & 0 & 0 & -7 \\
	0 & 0 & 1 & 0 & 1 \\
	0 & 0 & 0 & 1 & 3
	\end{array}\right]
\end{align}

\section{Homework Process and Study Group}

I did this homework by myself. 

\newpage

%\includepdf[pages=-]{prob*.pdf}

\end{document}
