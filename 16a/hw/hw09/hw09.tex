\documentclass[]{article}
\usepackage{amsmath}
\usepackage{amsfonts}
\usepackage{amssymb}
\usepackage{amsthm}
\usepackage{cancel}
\usepackage{graphicx}
\usepackage{siunitx}
\usepackage{circuitikz}
\usepackage{pdfpages}
\usepackage{hyperref}

\renewcommand{\thesection}{\arabic{section}}
\renewcommand{\thesubsection}{\thesection.\alph{subsection}}
\renewcommand{\thesubsubsection}{\thesubsection.\roman{subsubsection}}

\newtheorem{genthm}{Theorem}

%opening
\title{EECS 16A HW09}
\author{Bryan Ngo}
\date{2019-10-30}

\begin{document}

\maketitle

\section{Thévenin and Norton Equivalent Circuits}

In order to find \(V_{oc}\), we must find \(R_{eq}\), 
\begin{gather}
	R_{eq} = R_1 + (R_2 \parallel (R_3 + R_4)) = \SI{5}{\ohm} \\
	I_{eq} = \frac{V}{R_eq} = \SI{2}{\ampere}
\end{gather}
By KVL, we know that the voltage going through the \(R_2 \parallel (R_3 + R_4)\) branch must be \SI{4}{\volt}. 
Then, the current going through \(R_3 + R_4\) is \SI{0.66}{\ampere}. 
Then, the voltage is 
\begin{equation}
	V_{oc} = \SI{0.66}{\ampere} \cdot R_3 = \SI{3}{\volt}
\end{equation}
Now, finding \(I_{sc}\), we short over \(R_3\). Finding the equivalent current, 
\begin{gather}
	R_{eq} = R_1 + (R_2 \parallel R_4) = \SI{4}{\ohm} \\
	I_{eq} = \frac{V}{R_{eq}} = \SI{2.5}{\ampere}
\end{gather}
By KVL, the voltage of the parallel branch is \SI{2.5}{\volt}. 
Then, the current is
\begin{gather}
	I_{sc} = \frac{V_{branch}}{R_4} = \SI{1.66}{\ampere} \\
	R_{Th} = R_{N} = \frac{V_{oc}}{I_{sc}} = \SI{1,8}{\ohm}
\end{gather}
The Thévenin and Norton equivalents are
\begin{center}
\begin{circuitikz}[american]
	\draw (0, 3) to[V_=\SI{3}{\volt}] (0, 0);
	\draw (0, 3) to[R=\SI{1.8}{\ohm}] (3, 3);
	\draw (3, 3) node[ocirc]{\, A};
	\draw (0, 0) to[short] (3, 0) node[ocirc]{\, B};
\end{circuitikz} \hfill
\begin{circuitikz}[american]
	\draw (0, 0) to[I_=\SI{1.66}{\ampere}] (0, 3);
	\draw (2, 0) to[R=\SI{1.8}{\ohm}] (2, 3);
	\draw (0, 3) to[short] (3, 3) node[ocirc]{\, A};
	\draw (0, 0) to[short] (3, 0) node[ocirc]{\, B};
\end{circuitikz}
\end{center}

\section{Superposition}

Nulling the current source, 
\begin{center}
\begin{circuitikz}[american]
	\draw (2, 0) to[V=\SI{5}{\volt}] (0, 0);
	\draw (2, 0) to [R=\SI{3}{\ohm}] (4, 0);
	\draw (4, 0) to[short] (4, 2);
	\draw (0, 2) to[R=\SI{3}{\ohm},i=\(i\)] (4, 2);
	\draw (0, 0) to[short] (0, 2);
\end{circuitikz}
\end{center}
The current is simply
\begin{equation}
	i = I_{eq} = \frac{V}{R_{eq}} = \SI{0.833}{\ampere}
\end{equation}
Nulling the voltage source, 
\begin{center}
\begin{circuitikz}[american]
	\draw (0, 0) to[R=\SI{3}{\ohm}] (6, 0);
	\draw (6, 0) to[short] (6, 4);
	\draw (6, 4) to[R=\SI{3}{\ohm}, i<_=\(i\)] (0, 4);
	\draw (0, 4) to[short] (0, 0);
	\draw (0, 2) to[R=\SI{1.5}{\ohm}] (2, 2);
	\draw (4, 2) to[I= \SI{1}{\ampere}] (2, 2);
	\draw (4, 2) to[R=\SI{1.5}{\ohm}] (6, 2);
\end{circuitikz}
\end{center}
Since the circuit is symmetrical along the axis of current flow, it is easy to deduce that half of the current source current constitute \(i\), or \(i = \SI{0.5}{\ampere}\). \\
\\
By superposition, the total current is \(\sum i = \SI{1.333}{\ampere}\). 

\section{Charge Sharing}

\subsection{}

\begin{center}
\begin{circuitikz}[american, scale=0.6]
	\draw (0, 4) to[V=\(V_s\)] (0, 0);
	\draw (0, 0) node[ground]{};
	\draw (0, 4) to[short] (4, 4);
	\draw (4, 4) to[switch, l=\(\phi_1\)] (8, 4);
	\draw (8, 4) to[switch, l=\(\phi_2\)] (12, 4);
	\draw (12, 4) to[short] (16, 4) node[circ]{\, \(V_x\)};
	\draw (16, 4) to[switch, l=\(\phi_1\)] (16, 0) node[ground]{};
	\draw (4, 4) to[short] (4, 0);
	\draw (8, 4) to[C=\(C\), v=\(\)] (8, 0);
	\draw (12, 4) to[C=\(9C\), v=\(\)] (12, 0) node[ground]{};
	\draw (4, 0) to[switch, l=\(\phi_2\)] (8, 0);
	\draw (8, 0) to[switch, l=\(\phi_1\)] (12, 0);
\end{circuitikz}
\end{center}

\subsection{}

\begin{center}
\begin{circuitikz}[american, scale=0.6]
	\draw (0, 4) to[V=\(V_s\)] (0, 0);
	\draw (0, 0) node[ground]{};
	\draw (0, 4) to[short] (4, 4);
	\draw (4, 4) to[short, l=\(\phi_1\)] (8, 4);
	\draw (12, 4) to[short] (16, 4) node[circ]{\, \(V_x\)};
	\draw (16, 4) to[short, l=\(\phi_1\)] (16, 0) node[ground]{};
	\draw (8, 4) to[C=\(C\), v=\(\)] (8, 0);
	\draw (12, 4) to[C=\(9C\), v=\(\)] (12, 0) node[ground]{};
	\draw (8, 0) to[short, l=\(\phi_1\)] (12, 0);
\end{circuitikz}
\end{center}

\subsection{}

Finding the equivalent capacitance, 
\begin{equation}
	C_{eq} = C \parallel 9C = \frac{9}{10} C
\end{equation}
The charge on the equivalent capacitor, an thus the separate capacitors is
\begin{equation}
	Q_{eq} = \frac{9}{10} C V_{s}
\end{equation}
The voltage on \(C_1\) and \(C_2\)is 
\begin{align}
	V_{C_1} &= \frac{Q_{eq}}{C} = \frac{9}{10} V_s \\
	V_{C_2} &= \frac{Q_{eq}}{9C} = \frac{1}{10} V_s
\end{align}

\subsection{}

\begin{center}
\begin{circuitikz}[american, scale=0.6]
	\draw (0, 4) to[V=\(V_s\)] (0, 0);
	\draw (0, 0) node[ground]{};
	\draw (0, 4) to[short] (4, 4);
	\draw (8, 4) to[short, l=\(\phi_2\)] (12, 4);
	\draw (12, 4) to[short] (16, 4) node[circ]{\, \(V_x\)};
	\draw (4, 4) to[short] (4, 0);
	\draw (8, 4) to[C=\(C\), v=\(\)] (8, 0);
	\draw (12, 4) to[C=\(9C\), v=\(\)] (12, 0) node[ground]{};
	\draw (4, 0) to[short, l=\(\phi_2\)] (8, 0);
\end{circuitikz}
\end{center}

\(V_x\) is nothing more than the voltage across the capacitor, since it can be expressed as the node voltage relative to ground. 
We can determine the charge on the equivalent capacitor, and thus the separate capacitors, with KVL, 
\begin{align}
	V_s + V_{C_{eq}} &= 0 \\
	V_s + \frac{10 Q_{eq}}{9 C} &= 0 \\
	Q_{eq} &= -\frac{9}{10}C V_s
\end{align}
Then, the voltage on \(V_x\) is simply
\begin{equation}
	V_x = \frac{Q_{eq}}{9C} = -\frac{1}{10}V_s
\end{equation}

\section{Dynamic Random Access Memory}

If we assume constant current leakage and capacitance, 
\begin{align}
	Q &= CV \\
	\frac{dQ}{dt} = I &= C \frac{dV}{dt} \\
	\int_{0}^{t} \frac{I}{C} \, dt &= \int_{0}^{t} dV \\
	V(t) &= V(0) + \frac{I}{C} t
\end{align}
where \(V(0) = \SI{1.2}{\volt}, C = \SI{28d-15}{\farad}\). 
If we set \(t = \SI{1}{\milli\second}, V(t) = \SI{0.9}{\volt}\), we can solve for I, 
\begin{equation}
	I = C \frac{V(t) - V(0)}{t} = (\SI{28d-15}{\farad}) \left(\frac{0.9 \ \si{\volt} - 1.2 \ \si{\volt}}{0.001 \ \si{\second}}\right) = \SI{-8.4d-12}{\ampere}
\end{equation}

\section{Current Sources and Capacitors}

Since the capacitors are in series with a current source, the charge on both capacitors is simply \(Q = I_s t\). 
Then, the voltage across \(C_2\) is 
\begin{equation}
	V = \frac{I_s t}{C_2}
\end{equation}

\section{Super-Capacitors}

\subsection{}

\begin{center}
\begin{circuitikz}[american]
	\draw (0, 0) to[C, v=\(\)] (0, 2);
	\draw (0, 2) to[short] (2, 2);
	\draw (2, 0) to[I=\(i_{load}\)] (2, 2);
	\draw (2, 0) to[short] (0, 0);
\end{circuitikz} \hspace{2 mm}
\begin{circuitikz}[american]
	\draw (0, 0) to[C, v=\(\)] (0, 2);
	\draw (0, 2) to[C, v=\(\)] (2, 2);
	\draw (2, 0) to[I_=\(i_{load}\)] (2, 2);
	\draw (2, 0) to[short] (0, 0);
\end{circuitikz}
\begin{circuitikz}[american]
	\draw (0, 0) to[C, v=\(\)] (0, 2);
	\draw (0, 2) to[short] (4, 2);
	\draw (4, 0) to[I_=\(i_{load}\)] (4, 2);
	\draw (4, 0) to[short] (0, 0);
	\draw (2, 0) to[C, v=\(\)] (2, 2);
\end{circuitikz}
\end{center}

\subsection{}

Using the derivation from Problem 4, and taking note of the orientation of the load, 
\begin{align}
	V_1(t) &= v_{init} - \frac{i_{load}}{C_{sc}}t \\
	V_2(t) &= v_{init} - \frac{2 i_{load}}{C_{sc}}t \\
	V_3(t) &= v_{init} - \frac{i_{load}}{2C_{sc}}t
\end{align}

\subsection{}

At minimum, 
\begin{align}
	v_{min} &= v_{init} - \frac{i_{load}}{C_{sc}}t \\
	&= v_{init} - \frac{2 i_{load}}{C_{sc}}t \\
	&= v_{init} - \frac{i_{load}}{2C_{sc}}t
\end{align}
which means
\begin{align}
	t &= \frac{C_{sc}}{i_{load}}(v_{init} - v_{min}) \\
	&= \frac{C_{sc}}{2i_{load}}(v_{init} - v_{min}) \\
	&= \frac{2C_{sc}}{i_{load}}(v_{init} - v_{min})
\end{align}

\subsection{}

Configuration 3 is better than configuration 2 whenever \(v_{init} > v_{min}\). 
Observing the equations tells us that when this is satisfied, Config 3 has \emph{four times} the lifetime of Config 2. 

\subsection{}

We can express the change in energy of the capacitor as 
\begin{equation}
	\Delta E = E - E_0 = \frac{1}{2} (2C_{sc}) v_{min}^2 - \frac{1}{2} (2C_{sc}) v_{init}^2 = C_{sc} (v_{min}^2 - v_{init}^2)
\end{equation}
We set the final energy to be at \(v_{min}\) rather than \(0\) because that is the functional range of the capacitor. 

\section{It's Finally Raining!}

\subsection{}

When the tank is full, the capacitance is 
\begin{equation}
	C = 81 \epsilon \frac{h_{tot} \cancel{w}}{\cancel{w}} = 81 \epsilon \cdot h_{tot}
\end{equation}

\subsection{}

The capacitance of the water and air portions are
\begin{align}
	C_{water} &= 81 \epsilon \cdot h_{h2o} \\
	C_{air} &= \epsilon (h_{tot} - h_{h2o}) \\
	C_{tank} &= C_{air} + C_{water} = \epsilon (80 \cdot h_{h2o} + h_{tot})
\end{align}

\subsection{}

Once more, we use the equation
\begin{equation}
	V_C(t) = \frac{I_s}{C_{tank}}t = \frac{I_C}{C_{tank}}t
\end{equation}
There is no initial voltage since \(V(0) = \SI{0}{\volt}\). 

\subsection{}

Ideally, the plot of the voltage-time graph should be a line. 
Since we know that \(I_C\) is constant, the capacitance is simply the amount by which the current is attentuated through the graph. 
From there, we use the equation from Problem 4.b to solve for the height. 

\subsection{}

In the case of a comparator, its output is 
\begin{equation}
	V_{comp} = 
	\begin{cases}
	1 & V_{C} > V_{ref} \\
	-1 & \text{otherwise}
	\end{cases}
\end{equation}
This means \(T_1\) occurs at the exact moment
\begin{equation}
	V_{ref} = \frac{I_C}{C} T_1 \Longrightarrow C = \frac{I_C}{V_{ref}}T_1
\end{equation}

\stepcounter{section}
\stepcounter{section}

\section{Homework Process and Study Group}

I did this homework by myself. 

\newpage

%\includepdf[pages=-]{prob*.pdf}

\end{document}
